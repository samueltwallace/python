
\begin{xmlfeed}
\xmlfeedtitle{Slate Star Codex}
\begin{xmlfeedtitledetail}
\xmltitledetailtype{text/plain}\xmltitledetailbase{https://slatestarcodex.com/feed/}\xmltitledetailvalue{Slate Star Codex}
\end{xmlfeedtitledetail}

\begin{xmlfeedlinks}
\xmllinkshref{https://slatestarcodex.com/feed/}\xmllinksrel{self}\xmllinkstype{application/rss+xml}\xmllinksrel{alternate}\xmllinkstype{text/html}\xmllinkshref{https://slatestarcodex.com}
\end{xmlfeedlinks}
\xmlfeedlink{https://slatestarcodex.com}\xmlfeedsubtitle{SELF-RECOMMENDING!}
\begin{xmlfeedsubtitledetail}
\xmlsubtitledetailtype{text/html}\xmlsubtitledetailbase{https://slatestarcodex.com/feed/}\xmlsubtitledetailvalue{SELF-RECOMMENDING!}
\end{xmlfeedsubtitledetail}
\xmlfeedupdated{Fri, 10 Apr 2020 15:46:25 +0000}\xmlfeedlanguage{en-US}\xmlfeedsyupdateperiod{hourly}\xmlfeedsyupdatefrequency{1}
\begin{xmlfeedgeneratordetail}
\xmlgeneratordetailname{https://wordpress.org/?v=5.3.2}
\end{xmlfeedgeneratordetail}
\xmlfeedgenerator{https://wordpress.org/?v=5.3.2}
\end{xmlfeed}

\begin{xmlentries}
\xmlentriestitle{Coronalinks 4/10: Second Derivative}
\begin{xmlentriestitledetail}
\xmltitledetailtype{text/plain}\xmltitledetailbase{https://slatestarcodex.com/feed/}\xmltitledetailvalue{Coronalinks 4/10: Second Derivative}
\end{xmlentriestitledetail}

\begin{xmlentrieslinks}
\xmllinksrel{alternate}\xmllinkstype{text/html}\xmllinkshref{https://slatestarcodex.com/2020/04/10/coronalinks-4-10-second-derivative/}
\end{xmlentrieslinks}
\xmlentrieslink{https://slatestarcodex.com/2020/04/10/coronalinks-4-10-second-derivative/}\xmlentriescomments{https://slatestarcodex.com/2020/04/10/coronalinks-4-10-second-derivative/#comments}\xmlentriespublished{Fri, 10 Apr 2020 10:43:00 +0000}
\begin{xmlentriesauthors}
\xmlauthorsname{Scott Alexander}
\end{xmlentriesauthors}
\xmlentriesauthor{Scott Alexander}
\begin{xmlentriesauthordetail}
\xmlauthordetailname{Scott Alexander}
\end{xmlentriesauthordetail}

\begin{xmlentriestags}
\xmltagsterm{Uncategorized}\xmltagsterm{coronavirus}
\end{xmlentriestags}
\xmlentriesid{https://slatestarcodex.com/?p=5930}\xmlentriessummary{The second derivative is the rate of growth of the rate of growth. Over the past few weeks, the second derivative of total coronavirus cases switched from positive (typical of exponential growth) to zero or negative (typical of linear or &#8230; <a href="https://slatestarcodex.com/2020/04/10/coronalinks-4-10-second-derivative/">Continue reading <span class="pjgm-metanav">&#8594;</span></a>}
\begin{xmlentriessummarydetail}
\xmlsummarydetailtype{text/html}\xmlsummarydetailbase{https://slatestarcodex.com/feed/}\xmlsummarydetailvalue{The second derivative is the rate of growth of the rate of growth. Over the past few weeks, the second derivative of total coronavirus cases switched from positive (typical of exponential growth) to zero or negative (typical of linear or &#8230; <a href="https://slatestarcodex.com/2020/04/10/coronalinks-4-10-second-derivative/">Continue reading <span class="pjgm-metanav">&#8594;</span></a>}
\end{xmlentriessummarydetail}

\begin{xmlentriescontent}
\xmlcontenttype{text/html}\xmlcontentbase{https://slatestarcodex.com/feed/}\xmlcontentvalue{<p>The second derivative is the rate of growth of the rate of growth. Over the past few weeks, the second derivative of total coronavirus cases switched from positive (typical of exponential growth) to zero or negative (typical of linear or sublinear growth) in most European countries. Over the past few days, it switched from positive to zero/negative in the United States and the world as a whole. These are graphs of the rate of growth &#8211; notice how they go from shooting upward to being basically horizontal or downward-sloping (<A HREF="https://ourworldindata.org/coronavirus">source</A>).</p>
<p><center><IMG SRC="http://slatestarcodex.com/blog_images/derivative1.png"><br />
<IMG SRC="http://slatestarcodex.com/blog_images/derivative3.png"><br />
<IMG SRC="http://slatestarcodex.com/blog_images/derivative2.png"></center></p>
<p>This graph shows the numbers a little differently, (<A HREF="https://twitter.com/ScottGottliebMD/status/1246578533687836673/photo/1">source</A>), but you can see the same process going on in individual US cities:</p>
<p><center><IMG SRC="http://slatestarcodex.com/blog_images/gottleibgraph.jpg"></center></p>
<p>It would be premature to say we&#8217;re now winning the war on coronavirus. But we&#8217;ve stopped actively losing ground. If we were going to win, our first sign would be something like this. Current containment strategies are working. </p>
<p>As before, feel free to treat this as an open thread for all coronavirus-related issues. Everything here is speculative and not intended as medical advice.</p>
<p><b>The Bat Flu</b></p>
<p>SSC reader Trevor Klee has a great article on <A HREF="https://get21stnight.com/2020/03/30/why-do-we-keep-getting-diseases-from-bats/">why humans keep getting diseases from bats</A> (eg Ebola, SARS, Marburg virus, Nipah virus, coronavirus). He explains that because bats expend so much energy flying, they run higher body temperatures than other mammals, which degrades their DNA. Their DNA is such a mess that the usual immune system strategy of targeting suspicious DNA doesn&#8217;t work, so they accept constant low-grade infection with a bunch of viruses as a cost of doing business. Sometimes those viruses cross to humans, and then we get another bat-borne disease.</p>
<p>Subreddit user nodding_and_smiling doesn&#8217;t quite buy it</A>:</p>
<blockquote><p>I don&#8217;t think deep-diving into the bat immune system, while certainly very interesting, is necessary to explain the number zoonotic diseases from bats. I think a more important point is there is just a crazy number of bats, and the post doesn&#8217;t seem to fully appreciate this.</p>
<p>There are over 1,250 bat species in existence. This is about one fifth of all mammal species. Just to get a sense of this, let me ask a modified version of the question in the title:</p>
<p>&#8220;Why do human beings keep getting viruses from cows, sheep, horses, pigs, deer, bears, dogs, seals, cats, foxes, weasels, chimpanzees, monkeys, hares, and rabbits?&#8221;</p>
<p>That list contains species from four major mammal clades: ungulates (257 species), carnivora (270), primates (~300), and lagomorphs (91). Adding all these together, we don&#8217;t even get to 3/4 of the total number of bat species&#8230;</p></blockquote>
<p>Read <A HREF="https://www.reddit.com/r/slatestarcodex/comments/frshs0/why_do_human_beings_keep_getting_viruses_from_bats/flybjpq/">the full comment</A> (and the ensuing discussion) for more, including whether biodiversity vs raw numbers is the appropriate measure here.</p>
<p><b>Mail Suffrage</b></p>
<p>The Wisconsin Democratic primary (plus some unrelated elections) <A HREF="https://fivethirtyeight.com/features/election-preview-yes-wisconsin-is-still-holding-its-primary-on-tuesday/">went ahead as usual</A> this week, with people going out to voting booths instead of voting by mail. Democrats wanted to allow (mandate?) mail voting, but Republicans refused.</p>
<p>Presumably Republicans assumed mail voting would benefit Democrats? The last time a state instituted vote-by-mail, in New Jersey, it did seem to <A HREF="https://www.politico.com/states/new-jersey/story/2019/10/24/democrats-benefiting-from-early-surge-in-vote-by-mail-ballots-1225915">increase the Democratic share of the vote</A>.</p>
<p>I&#8217;m surprised by this, because I would have expected mail voting, as opposed to booth voting, benefits people with good executive function who are familiar with doing things by mail &#8211; ie older, richer people, ie Republicans. It would appear that I am wrong.</p>
<p>What if the epidemic isn&#8217;t done by November? There will probably be a discussion of lifting the shutdown to have a normal election, vs. voting entirely by mail, vs. combination where people who want to vote by mail can but the polls are open for everyone else. I don&#8217;t know if the second option is in the Overton Window right now (or if it should be). The party lines here seem to be the same: Nancy Pelosi is <A HREF="https://twitter.com/SpeakerPelosi/status/1246203557772230659">already pushing for it</A>, and conservatives are <A HREF="https://twitter.com/ScottTParkinson/status/1242555770018947077">already denouncing it as a liberal plot</A>.</p>
<p>I&#8217;m in favor, obviously, but also terrified that something goes wrong. In one scenario, failure to agree on vote-by-mail rules (or failure to implement them competently)  delays the election, with no clear way to get it back on track. In another, the sudden panicked switch to a less-tested voting method goes wrong in unpredictable ways and creates ambiguity over election results. It <A HREF="https://www.politico.com/news/magazine/2020/04/07/danger-moving-vote-by-mail-168602">could be</A> Bush v. Gore x 1000.</p>
<p>The Neoliberal Project has <A HREF="https://exponents.substack.com/p/let-everyone-vote-by-mail">an analysis</A> of what we should do and how to make postal voting work. I just really hope it doesn&#8217;t come to this.</p>
<p><b>Charity Update</b></p>
<p>Last week I linked a list of potentially good coronavirus charities cobbled together by some random people on the EA forum. Now a more serious organization, 80,000 Hours, <A HREF="https://80000hours.org/articles/covid-19-options-for-donating/">has posted their own list</A>.</p>
<p>The top option is still the Johns Hopkins Center for Health Security, which researches and advocates for biosecurity policy. Last week someone in the comments doubted the quality of their work, pointing out that one of their flagship efforts is a <A HREF="https://www.ghsindex.org/">ranking of how prepared different countries are for a global pandemic</A>; their 2019 listing put the US at the top, which now feels like a cruel joke. But I&#8217;m not sure how much to hold it against them. Looking at their webpage, it mostly investigates whether a country has good plans addressing various issues of a crisis, and lots of resources that it can deploy if needed. As best I can tell, the US had great plans and didn&#8217;t follow any of them, and lots of resources which it totally failed to deploy effectively. Responsible think tanks are probably not allowed to add a -10000 points at the end of their analysis for &#8220;but its leaders are idiots&#8221;. This might still be a good time to reread <A HREF="https://samzdat.com/2018/03/26/enter-a-search-term-e-g-democracy/">Samzdat on hokey country rankings</A> and <A HREF="https://deponysum.com/2019/06/06/economic-freedom-indexes-are-bad-actually/">no_bear_so_low on hokey country rankings</A>.</p>
<p>Speaking of charity, you can read on Twitter about <A HREF="https://twitter.com/MsMelChen/status/1245501506205421570">the trials and tribulations of people</A> trying to donate face masks to hospitals, and here&#8217;s an article from three years ago about <A HREF="https://www.worldbank.org/en/news/press-release/2017/06/28/world-bank-launches-first-ever-pandemic-bonds-to-support-500-million-pandemic-emergency-financing-facility">issuing pandemic bonds</A> as a novel insurance-type way of funding global disease response. Pretty neat.</p>
<p>And you might think that a page called <A HREF="https://www.thecovidchallenge.org/">The COVID Challenge</A> where you sign up to deliberately get infected with coronavirus is a bad idea, but it&#8217;s actually some volunteers trying to make a list of people who would be willing to get deliberately infected (if it came to that) in order to test vaccines, which they will hand over to vaccine-makers once they get to the testing stage. Rationalist John Beshir <A HREF="https://www.vox.com/future-perfect/2019/2/21/18235136/vaccine-malaria-research-trials-volunteers">did something like this</A> for a malaria vaccine last year and earned $3200 (plus the warm glow of having made a difference) by letting himself getting bitten by infected mosquitoes in an Oxford laboratory.</p>
<p><b>There Is No Coronavirus In Ba Sing Se</b></p>
<p>Turkmenistan is a strange country. You probably remember it for its wacky former dictator Turkmenbashi, who <A HREF="https://slate.com/human-interest/2014/02/saparmurat-niyazov-former-president-of-turkmenistan-has-left-quite-the-legacy-in-ashgabat.html">among other things</A> renamed the month of March after his mother, and told citizens that anyone who read his book three times would enter Heaven. Or for its wacky current dictator Gurbanguly Berdymukhamedov, who NPR <A HREF="https://www.npr.org/sections/coronavirus-live-updates/2020/03/31/824611607/turkmenistan-has-banned-use-of-the-word-coronavirus">describes</A> as a &#8220;dentist/rapper/strongman&#8221;. Or for its impressive accomplishment of beating out North Korea to be named <A HREF="https://rsf.org/en/ranking">the most repressive country on Earth</A> by Reporters Without Borders. </p>
<p>Its coronavirus response will do nothing to improve its reputation: early reports claimed it had <A HREF="https://rsf.org/en/news/coronavirus-limits-turkmenistan">banned mentioning the word &#8216;coronavirus&#8217; or acknowledging its existence in any way</A>.  </p>
<p>The Diplomat <A HREF="https://thediplomat.com/2020/04/did-turkmenistan-really-ban-the-word-coronavirus/">argues this is not quite true</A>; some state media seems to be using the word. But they are definitely arresting people for talking about it outside official government organs, and they are definitely denying that there are any cases in the country. Since Turkmenistan is right next to Iran, which has had thousands of cases for months, this is pretty implausible.</p>
<p>The Diplomat also requests that people try not to focus on the country&#8217;s wacky dictators so much every time they talk about it, since that makes it hard to get people to take its suffering seriously. Sorry, Diplomat and Turkmen people 🙁</p>
<p>And SSC reader Castilho <A HREF="https://slatestarcodex.com/2020/03/27/coronalinks-3-27-20/#comment-872126">describes</A> their home country of Brazil, which seems to be right up there with Turkmenistan:</p>
<blockquote><p>We’re one of the few countries in the developing world that actually could handle the pandemic reasonably well (We have around 61.000 ventilators, or 1 ventilator per 3.300 people, which isn’t actually that bad and could be expanded for a decent epidemic response)&#8230;</p>
<p>However, our president has decided to go all-in on denying how serious the virus is. The Atlantic even called him “the new leader of the Coronavirus denial movement“. He’s accusing local politicians who have instituted lockdowns of plotting to destroy the country’s economy in order to use it against him later. His sons, who are local politicians in the wealthy parts of the country, have been saying this is all a plot by leftist politicians together with the People’s Republic of China to make him and Trump look bad. I wish I was kidding&#8230;</p>
<p>The worst part is that he’s led a nationwide movement telling people to leave their homes and go back to their normal lives. The government actually wanted to make “Brazil can’t stop” into a nationwide campaign, but when a significant part of the population didn’t appreciate it, they just deleted the social media posts and now they claim there never was such a campaign.</p></blockquote>
<p>Read <A HREF="https://slatestarcodex.com/2020/03/27/coronalinks-3-27-20/#comment-872126">the full comment</A> for more.</p>
<p>And last month I wondered about the surprisingly slow spread of cases in Iran. I can&#8217;t find anyone saying so outright, but it seems like the numbers are probably wrong. At least that&#8217;s what I gather from articles like <A HREF="https://www.dw.com/en/iran-faces-catastrophic-death-toll-from-coronavirus/a-52811895">this</A> and Twitter accounts like <A HREF="https://twitter.com/aliostad">this</A> highlighting the scale of the crisis there, which seems at least as bad as anywhere in the world. I don&#8217;t know if they&#8217;re deliberately lying about case numbers (why start now, after the numbers were so bad a few weeks ago?) or if testing has just completely broken down there. See also <A HREF="https://www.nytimes.com/2020/03/17/world/middleeast/coronavirus-iran-rouhani.html">this article</A> on how their form of government has led to power struggles and a garbled response. I would say something mean about radical Islamic fundamentalism, except that the whole thing mirrors blow for blow what happened between Cuomo and de Blasio in New York.</p>
<p>And finally, here&#8217;s a great article <A HREF="https://medium.com/@cuttlefishification/the-curious-case-of-japan-in-the-covid-times-where-it-all-went-wrong-2d0194b6d779">on the mystery of Japan</A>. Tl;dr: cultural traditions like mask-wearing and bowing helped it for a while, crowded trains aren&#8217;t as bad as you&#8217;d think because nobody&#8217;s talking, banning large gatherings very early was a <i>really</i> good move, their weak half-hearted version of test-and-trace worked for a while out of sheer luck, but now cases are finally starting to rise and there probably won&#8217;t be a mystery to explain for much longer.</p>
<p><b>Economic Unanimity</b></p>
<p>The IGM Economics Experts Panel surveys a view dozen top economists on the issues of the day. This month they&#8217;re focusing on coronavirus. Here are some sample results:</p>
<p><center><IMG SRC="http://slatestarcodex.com/blog_images/igm_covid.png"></center></p>
<p>&#8230;they pretty unanimously support the lockdown, even when asked only to reflect on its economic impact.</p>
<p>Some socialists on social media are trying to spread a narrative where capitalists think the economy is more important than lives and want to lift the lockdown immediately, and it&#8217;s only socialists who are standing up for the importance of saving people. Top economists aren&#8217;t a perfect stand-in for capitalists, but it&#8217;s still pretty clear that they&#8217;re wrong.</p>
<p>Also, there are starting to be some econ papers trying to more rigorously analyze the pros and cons of lockdown. <A HREF="https://papers.ssrn.com/sol3/papers.cfm?abstract_id=3561934">The Benefits and Costs of Flattening the Curve for COVID-19</A> says that &#8220;assuming that social distancing measures can substantially reduce contacts among individuals, we find net benefits of roughly $5 trillion in our benchmark scenario&#8221;.</p>
<p><b>USA! USA!</b></p>
<p>Is there anything Americans can be proud of here?</p>
<p><A HREF="https://twitter.com/Noahpinion/status/1240348644554887168">@noahpinion reminds us</A> of America&#8217;s long history of being late on the trigger but doing a great job once we get started (Churchill: &#8220;You can always count on Americans to do the right thing &#8211; after they&#8217;ve tried everything else.&#8221;). We were late entrants into both World Wars but had an outsized effect on both of them. In that spirit, although we were very slow to start testing, we&#8217;ve ramped up impressively fast &#8211; from almost none to 1/3 of South Korean levels per capita within a few weeks. </p>
<p>Also worth celebrating &#8211; during the Wuhan phase of the pandemic, China <A HREF="https://www.nytimes.com/2020/02/03/world/asia/coronavirus-wuhan-hospital.html">built an impromptu 1,000 patient hospital in ten days</A>. US media reported this as unbelievable &#8211; a sign that a young and vigorous country could accomplish feats that a decadent America could never dream of. But last week in New York, the Army Corps of Engineers converted the Javits Convention Center into an impromptu 2,000 patient hospital in&#8230;<A HREF="https://www.defense.gov/Explore/News/Article/Article/2133514/corps-of-engineers-converts-nycs-javits-center-into-hospital/">about ten days</A>. </p>
<p>I don&#8217;t know, maybe this was easier because they&#8217;re converting an existing structure instead of building a whole new one (though even the Chinese used prefab units). But it&#8217;s nice to know we still have it in us to do things quickly. There&#8217;s no civilizational decline. If the government ever legalized building things quickly again, we&#8217;d be mopping the floor with China within weeks.</p>
<p><b>Legal Immunity</b></p>
<p>There&#8217;s a Jewish legal principle called <A HREF="https://en.wikipedia.org/wiki/Marit_ayin"><i>marit ayin</i></A>, which means that it&#8217;s illegal to do something which is legal but looks illegal. For example, you can&#8217;t eat some kind of plant-based Impossible Bacon, because it would look like you were eating real bacon. Some authorities say it is sometimes permissible to eat the Impossible Bacon if you leave the box out in a prominent position so that it doesn&#8217;t look illegal; I&#8217;m not sure of the details.</p>
<p>The argument is that widespread flagrant unpunished violation of the law makes the law uncompelling and unenforceable, and this is true whether the violation is real or imagined. If you never see anyone eat bacon, you probably won&#8217;t eat it yourself; if everyone around you seems to be eating bacon all the time, it feels less taboo. Also, if you&#8217;re a police officer, it&#8217;s hard to identify the real bacon eaters if there are a bunch of people eating Impossible Bacon who get annoyed every time you question them.</p>
<p>I was thinking about this recently with the news that Germany is considering issuing <A HREF="https://www.businessinsider.com/coronavirus-germany-covid-19-immunity-certificates-testing-social-distancing-lockdown-2020-3">immunity certificates</A> for people who have gotten coronavirus, recovered, and are now safe to do normal activities. It&#8217;s a good idea, but suffers from the same problem as Impossible Bacon &#8211; if there are hundreds of people going outside maskless, eating at restaurants, and sunning themselves on the beach, it&#8217;s going to be hard for the rest of us to take lockdown seriously enough.</p>
<p>The equivalent of the rabbis&#8217; put-the-box-out solution would be for governments to issue not just a certificate but some kind of unique article of clothing people could wear to mark their status. For example, they might give an unusually shaped red cap &#8211; if the beaches are full of people in red caps, that&#8217;s fine and doesn&#8217;t say anything about whether you personally should go sunbathe. And if the beachgoers see someone without a red cap, they can question them or keep their distance.</p>
<p>This would take a lot of centralized coordination, though. I&#8217;m not sure how you could send the same message without a government order explaining what the cap meant to everybody. Though (as per <A HREF="https://www.theonion.com/we-have-coronavirus-under-control-announces-cdc-dire-1842128516">this Onion article</A>) wearing a fake pangolin snout over your nose would send a strong signal.</p>
<p>A reader who has overcome the disease emailed me to ask whether there are any useful volunteer opportunities for people like him &#8211; anyone have any advice?</p>
<p><b>Short Links</b></p>
<p>Last week I expressed confusion about how to measure population density so that arbitrary choices of border don&#8217;t distort the results. Commenters delivered by finding me this article on <A HREF="https://www.citylab.com/equity/2012/10/americas-truly-densest-metros/3450/">population-weighted density</A>, which solves my theoretical concerns but doesn&#8217;t really change any of the numbers much.</p>
<p>The Netherlands is another country which, like Sweden and Brazil, is <A HREF="https://www.bbc.com/news/world-europe-52135814">volunteering to be the control group</A> for the great experiment of whether national lockdowns work. Maybe someone should compare them to Belgium or somewhere like that in a few months and see how they did.</p>
<p>An aircraft carrier captain publicly complained that the Navy was failing to address an epidemic aboard his ship; the Navy <A HREF="https://www.reuters.com/article/us-health-coronavirus-usa-navy-confirmat/u-s-navy-relieves-commander-of-coronavirus-hit-aircraft-carrier-idUSKBN21K3EC?il=0">fired him for whistleblowing</A>. I&#8217;m having a hard time thinking of any perspective other than &#8220;the Navy is bad and should be torn down totally to the foundations, preferably using some sort of land-based weapon so they can&#8217;t fight back&#8221;, but <A HREF="https://twitter.com/ConsWahoo/status/1246079067637587968">here&#8217;s a different ex-captain</A> trying his best to give a nuanced perspective. </p>
<p>Say what you will about the New York Times&#8217; coverage lately, but their <A HREF="https://pbs.twimg.com/media/EUIr0ffX0AE3lsf?format=jpg&#038;name=medium">cover design</A> remains second to none.</p>
<p><A HREF="https://comparativelysuperlative.tumblr.com/post/614509645957136384/invertedporcupine-fieldsoffire100-im-gonna">This Tumblr post</A> has a discussion of how/whether a Clinton administration might have responded differently to the pandemic, but the part I like is the discussion of the phrase &#8220;follow the pandemic response playbook&#8221;. It turns out this is a literal document, called the Playbook For Early Response To High Consequence Emerging Infectious Disease Threats And Biological Incidents, and you can read it <A HREF="https://www.politico.com/news/2020/03/25/trump-coronavirus-national-security-council-149285">here</A>. </p>
<p>Marginal Revolution: <A HREF="https://slatestarcodex.com/2020/04/05/open-thread-151/#comment-876142">are hospitals really saving that many people?</A></p>
<p>UK clinical guideline body NICE now officially <A HREF="https://www.bmj.com/content/369/bmj.m1409">recommends against using NSAIDs for coronavirus</A>. Still not completely proven, but I think they&#8217;re right to advise caution. While most experts themselves behaved appropriately, this is more egg on the face of the media, which until a few weeks ago was <A HREF="https://www.latimes.com/science/story/2020-03-18/theres-no-good-reason-to-avoid-ibuprofen-if-you-are-infected-with-the-coronavirus">running</A> <A HREF="https://www.buzzfeednews.com/article/stephaniemlee/coronavirus-ibuprofen-who">stories</A> telling people this was a myth and they should ignore it.</p>
<p>538 surveyed infectious disease experts around the US, asking them to predict the number of cases in X days&#8217; time, with confidence intervals. The results are in, and <A HREF="https://twitter.com/katy_milkman/status/1244668082062348291">the experts did worse than just continuing the exponential curve on the graph would have</A>. EDIT: But see <A HREF="https://www.reddit.com/r/slatestarcodex/comments/fye0hr/coronalinks_410_second_derivative/fmzmkih/">here</A>.</p>
<p>If you&#8217;re following Robin Hanson&#8217;s variolation proposals, you can watch Hanson debate <A HREF="https://www.lesswrong.com/posts/eWdh7pjMRKiF8Jzwt/hanson-and-mowshowitz-debate-covid-19-variolation">vs. Zvi Moskowitz<A HREF="https://twitter.com/robinhanson/status/1245423564066611206"> and <A HREF="https://twitter.com/robinhanson/status/1245423564066611206">vs. Greg Cochran</A> (and here&#8217;s <A HREF="https://marginalrevolution.com/marginalrevolution/2020/04/why-i-do-not-favor-variolation-for-covid-19.html">Cowen on Hanson</A>). Anyway, viral dose <A HREF="https://www.nytimes.com/2020/04/01/opinion/coronavirus-viral-dose.html">seems to have gone mainstream</A>, though nobody seems to be doing anything about it yet.</p>
<p>The <A HREF="https://twitter.com/NateSilver538/status/1248430886170722304">two different interpretations of &#8220;flatten the curve&#8221;</A>. I think this explains why so much of the discussion around this phrase has been confusing.</p>
<p><A HREF="https://globalnews.ca/news/6772979/coronavirus-3m-n95-respirators-trump-canada/">Trump Asks Medical Supply Firm 3M To Stop Selling N95 Respirators To Canada</A>, and also <A HREf="https://theintercept.com/2020/04/01/coronavirus-medical-supplies-export/">Key Medical Supplies Were Shipped From US Manufacturers To Foreign Buyers</A>. I think we&#8217;re supposed to be outraged about both of those things simultaneously but I can&#8217;t manage it, maybe some of you will have better luck.</p>
<p>How much risk do young people really face from coronavirus? What are the risks of long-term complications? <A HREF="https://srconstantin.github.io/2020/03/16/COVID-19-young-people-risks.html">Sarah C investigates</A>. </p>
<p>Last week, Elon Musk got widespread praise (including here) for donating a thousand ventilators he managed to procure through his Tesla supply chain. Now the picture has become more confusing. Reporters looking at a picture of his shipment noticed that <A HREF="https://ftalphaville.ft.com/2020/04/01/1585782924000/Elon-Musk-promised-ventilators--These-are-BPAP-machines-/">the boxes pictured are for BiPAP machines</A> &#8211; technically a kind of ventilator, but not the kind hospitals need to fight coronavirus. Was the whole thing a giant mistake or cynical PR stunt? But then some hospitals <A HREF="https://twitter.com/NYCHealthSystem/status/1246460768310214657">tweeted thanking Tesla</A> specifically for delivering &#8220;Medtronic invasive ventilators&#8221;, which <i>are</i> the kind hospitals need to fight coronavirus. Some people are theorizing that maybe hospitals don&#8217;t want to offend Musk since he might have real ventilators later, other people that maybe Musk got both some useful and some non-useful ventilators in his shipment. I dunno. In any case, he&#8217;s <A HREF="https://techcrunch.com/2020/04/05/tesla-shows-how-its-building-ventilators-with-car-parts/">still promising to make some at Tesla factories</A>, though.</p>}
\end{xmlentriescontent}
\xmlentrieswfwcommentrss{https://slatestarcodex.com/2020/04/10/coronalinks-4-10-second-derivative/feed/}\xmlentriesslashcomments{138}\xmlentriestitle{2019 Predictions: Calibration Results}
\begin{xmlentriestitledetail}
\xmltitledetailtype{text/plain}\xmltitledetailbase{https://slatestarcodex.com/feed/}\xmltitledetailvalue{2019 Predictions: Calibration Results}
\end{xmlentriestitledetail}

\begin{xmlentrieslinks}
\xmllinksrel{alternate}\xmllinkstype{text/html}\xmllinkshref{https://slatestarcodex.com/2020/04/08/2019-predictions-calibration-results/}
\end{xmlentrieslinks}
\xmlentrieslink{https://slatestarcodex.com/2020/04/08/2019-predictions-calibration-results/}\xmlentriescomments{https://slatestarcodex.com/2020/04/08/2019-predictions-calibration-results/#comments}\xmlentriespublished{Thu, 09 Apr 2020 04:22:00 +0000}
\begin{xmlentriesauthors}
\xmlauthorsname{Scott Alexander}
\end{xmlentriesauthors}
\xmlentriesauthor{Scott Alexander}
\begin{xmlentriesauthordetail}
\xmlauthordetailname{Scott Alexander}
\end{xmlentriesauthordetail}

\begin{xmlentriestags}
\xmltagsterm{Uncategorized}\xmltagsterm{predictions}
\end{xmlentriestags}
\xmlentriesid{https://slatestarcodex.com/?p=5928}\xmlentriessummary{At the beginning of every year, I make predictions. At the end of every year, I score them (this year I&#8217;m very late). Here are 2014, 2015, 2016, 2017, and 2018. And here are the predictions I made for 2019. &#8230; <a href="https://slatestarcodex.com/2020/04/08/2019-predictions-calibration-results/">Continue reading <span class="pjgm-metanav">&#8594;</span></a>}
\begin{xmlentriessummarydetail}
\xmlsummarydetailtype{text/html}\xmlsummarydetailbase{https://slatestarcodex.com/feed/}\xmlsummarydetailvalue{At the beginning of every year, I make predictions. At the end of every year, I score them (this year I&#8217;m very late). Here are 2014, 2015, 2016, 2017, and 2018. And here are the predictions I made for 2019. &#8230; <a href="https://slatestarcodex.com/2020/04/08/2019-predictions-calibration-results/">Continue reading <span class="pjgm-metanav">&#8594;</span></a>}
\end{xmlentriessummarydetail}

\begin{xmlentriescontent}
\xmlcontenttype{text/html}\xmlcontentbase{https://slatestarcodex.com/feed/}\xmlcontentvalue{<p>At the beginning of every year, I make predictions. At the end of every year, I score them (this year I&#8217;m very late). Here are <A HREF="http://slatestarcodex.com/2015/01/01/2014-predictions-calibration-results/">2014</A>, <A HREF="http://slatestarcodex.com/2016/01/02/2015-predictions-calibration-results/">2015</A>, <A HREF="http://slatestarcodex.com/2016/12/31/2016-predictions-calibration-results/">2016</A>, <A HREF="https://slatestarcodex.com/2018/01/02/2017-predictions-calibration-results/">2017</A>, and <A HREF="https://slatestarcodex.com/2019/01/22/2018-predictions-calibration-results/">2018</A>.</p>
<p>And here are the predictions I made for 2019. Strikethrough’d are false. Intact are true. Italicized are getting thrown out because I can’t decide if they’re true or not. All of these judgments were as of December 31 2019, not as of now.</p>
<p>Please don’t complain that 50% predictions don’t mean anything; I know this is true but there are some things I’m genuinely 50-50 unsure of. Some predictions are redacted because they involve my private life or the lives of people close to me. A few that started off redacted stopped being secret; I&#8217;ve put those in [brackets].</p>
<p>US<br />
1. Donald Trump remains President: 90%<br />
2. Donald Trump is impeached by the House: 40%<br />
<s>3. Kamala Harris leads the Democratic field: 20%</s><br />
<s>4. Bernie Sanders leads the Democratic field: 20%</s><br />
5. Joe Biden leads the Democratic field: 20%<br />
<s>6. Beto O’Rourke leads the Democratic field: 20%</s><br />
7. Trump is still leading in prediction markets to be Republican nominee: 70%<br />
8. Polls show more people support the leading Democrat than the leading Republican: 80%<br />
9. Trump’s approval rating below 50: 90%<br />
<s>10. Trump’s approval rating below 40: 50%</s><br />
<s>11. Current government shutdown ends before Feb 1: 40%</s><br />
12. Current government shutdown ends before Mar 1: 80%<br />
13. Current government shutdown ends before Apr 1: 95%<br />
<s>14. Trump gets at least half the wall funding he wants from current shutdown: 20%</s><br />
15. Ginsberg still alive: 50%</p>
<p>ECON AND TECH<br />
16. Bitcoin above 1000: 90%<br />
17. Bitcoin above 3000: 50%<br />
18. Bitcoin above 5000: 20%<br />
19. Bitcoin above Ethereum: 95%<br />
20. Dow above current value of 25000: 80%<br />
<s>21. SpaceX successfully launches and returns crewed spacecraft: 90%</s><br />
<s>22. SpaceX Starship reaches orbit: 10%</s><br />
23. No city where a member of the general public can ride self-driving car without attendant: 90%<br />
<s>24. I can buy an Impossible Burger at a grocery store within a 30 minute walk from my house: 70%</s><br />
25. Pregabalin successfully goes generic and costs less than $100/month on GoodRx.com: 50%<br />
26. No further CRISPR-edited babies born: 80%</p>
<p>WORLD<br />
<s>27. Britain out of EU: 60%</s><br />
<s>28. Britain holds second Brexit referendum: 20%</s><br />
29. No other EU country announces plan to leave: 80%<br />
30. China does not manage to avert economic crisis (subjective): 50%<br />
31. Xi still in power: 95%<br />
32. MbS still in power: 95%<br />
<s>33. May still in power: 70%</s><br />
34. Nothing more embarassing than Vigano memo happens to Pope Francis: 80%</p>
<p>SURVEY<br />
35. …finds birth order effect is significantly affected by age gap: 40%<br />
36. …finds fluoxetine has significantly less discontinuation issues than average: 60%<br />
37. …finds STEM jobs do not have significantly more perceived gender bias than non-STEM: 60%<br />
<i>38. …finds gender-essentialism vs. food-essentialism correlation greater than 0.075: 30%</i></p>
<p>PERSONAL &#8211; INTERNET<br />
39. SSC gets fewer hits than last year: 70%<br />
40. I finish and post [New Atheism: The Godlessness That Failed]: 90%<br />
<s>41. I finish and post [Structures Of Paranoia]: 50%</s><br />
42. I finish and post [a sequence based on Secret Of Our Success]: 50%<br />
43. [New Atheism] post gets at least 40,000 hits: 40%<br />
<s>44. [The Proverbial Murder Mystery] post gets at least 40,000 hits: 20%</s><br />
<s>45. New co-blogger with more than 3 posts: 20%</s><br />
46. Repeat adversarial collaboration contest with at least 5 entries: 60%<br />
47. [Culture War thread successfully removed from subreddit]: 90%<br />
48. [Culture War new version getting at least 500 comments per week]: 70%<br />
<s>49. I start using Twitter again (5+ tweets in any month): 60%</s><br />
50. I start using Facebook again (following at least 5 people): 30%</p>
<p>PERSONAL &#8211; HEALTH<br />
51. I get the blood tests I should be getting this year: 90%<br />
52. I try one biohacking project per month x at least 10 months: 30%<br />
53. I continue taking sceletium regularly: 70%<br />
<s>54. I switch from [Zembrin to Tristill] for at least 3 months: 20%</s><br />
<s>55. I find at least one new supplement I take or expect to take regularly x 3 months: 20%</s><br />
<s>56. Minoxidil use produces obvious progress: 50%</s><br />
<s>57. I restart [redacted]: 20%</s><br />
58. I spend one month at least substantially more vegetarian than my current compromise: 20%<br />
<s>59. I spend one month at least substantially less vegetarian than my current compromise: 30%</s><br />
60. I weigh more than 195 lbs at year end: 80%<br />
<s>61. I meditate at least 30 minutes/day more than half of days this year: 30%</s><br />
<s>62. I use marijuana at least once this year: 20%</s></p>
<p>PERSONAL &#8211; PROJECTS<br />
<s>63. I finish at least 10% more of [redacted]: 20%</s><br />
<s>64. I completely finish [redacted]: 10%</s><br />
<s>65. I finish and post [redacted]: 5%</s><br />
<s>66. I write at least ten pages of something I intend to turn into a full-length book this year: 20%</s><br />
<s>67. I practice calligraphy at least seven days in the last quarter of 2019: 40%</s><br />
<s>68. I finish at least one page of the [redacted] calligraphy project this year: 30%</s><br />
<s>69. I finish the entire [redacted] calligraphy project this year: 10%</s><br />
<s>70. I finish some other at-least-one-page calligraphy project this year: 80%</s></p>
<p>PERSONAL &#8211; PROFESSIONAL<br />
71. I attend the APA Meeting: 80%<br />
<s>72. [redacted]: 50%</s><br />
<i>73. [redacted]: 40%</i><br />
74. I still work in SF with no plans to leave it: 60%<br />
75. I still only do telepsychiatry one day with no plans to increase it: 60%<br />
76. I still work the current number of hours per week: 60%<br />
77. I have not started (= formally see first patient) my own practice: 80%<br />
78. I lease another version of the same car I have now: 90%</p>
<p>PERSONAL &#8211; HOUSE<br />
79. I still live in my current house with no specific plans to leave: 80%<br />
80. I set up a decent home library: 60%<br />
81. We got a second trash can: 90%<br />
82. The gate is fixed with no problems at all: 50%<br />
83. The ugly paint spot on my wall gets fixed: 30%<br />
84. There is some kind of nice garden: 60%<br />
85. &#8230;and I am at least half responsible: 20%<br />
<s>86. I get my own washing machine: 20%</s><br />
87. There is another baby in my house: 60%<br />
<i>88. No other non-baby resident (expected 6+ month) in my house who doesn&#8217;t live there now: 70%</i><br />
89. No existing resident moves away (except the one I already know about): 80%<br />
90. No other long-term (expected 6+ month) resident of my subunit who doesn&#8217;t live there now: 80%<br />
<i>91. [Decision Tree House] is widely considered a success: 70%</i><br />
<s>92. &#8230;with plans (vague okay) to create a second one: 20%</s></p>
<p>PERSONAL &#8211; ROMANCE<br />
<s>93. I find a primary partner: 30%</s><br />
94. I go on at least one date with someone who doesn’t already have a primary partner: 90%<br />
95. I remake an account on OKCupid: 80%<br />
<s>96. [redacted]: 10%</s><br />
<s>97. [redacted]: 20%</s><br />
<s>98. [redacted]: 20%</s><br />
<s>99. [redacted]: 20%</s><br />
100. [redacted]: 20%<br />
<s>101. [redacted]: 30%</s><br />
<s>102. [redacted]: 10%</s><br />
103. [redacted]: 30%<br />
<s>104. [I go on at least three dates with someone I have not yet met]: 50%</s><br />
<s>105. [redacted]: 10%</s><br />
106. [redacted]: 50%</p>
<p>PERSONAL &#8211; FRIENDS<br />
107. I am still playing D&#038;D: 60%<br />
108. I go on a trip to Guatemala: 90%<br />
<s>109. I go on at least one other international trip: 30%</s><br />
<s>110. I go to at least one Solstice outside the Bay: 40%</s><br />
111. I go to at least one city just for an SSC meetup: 30%<br />
112. [redacted] is in a relationship: 40%<br />
<s>113. [redacted] still has their current partner: 50%</s><br />
<s>114. [redacted] is at their current job: 20%</s><br />
115. [redacted] is still at their current job: 80%<br />
<s>116. I hang out with [redacted] at least once: 60%</s><br />
117. I hang out with [redacted] at least once: 60%<br />
118. I am in [redacted] Discord server: 80%</p>
<p><center><IMG SRC="http://slatestarcodex.com/blog_images/calibration2019.png"></p>
<p><i><font size="1">Calibration chart. The red line represents perfect calibration, the blue my predictions. The closer they are, the better I am doing.</font></i></center></p>
<p>Of 11 predictions at 50%, I got 4 wrong and 7 right, for an average of 64%<br />
Of 22 predictions at 60%, I got 7 wrong and 15 right, for an average of 68%<br />
Of 17 predictions at 70%, I got 5 wrong and 12 right, for an average of 71%<br />
Of 37 predictions at 80%, I got 6 wrong and 31 right, for an average of 83%<br />
Of 17 predictions at 90%, I got 1 wrong and 16 right, for an average of 94%<br />
Of 5 predictions at 95%, I got 0 wrong and 5 right, for an average of 100%</p>
<p>50% predictions are technically meaningless since I could have written them either way. I&#8217;ve lightened them on the chart to indicate they can be ignored.</p>
<p>It was another good year for me. Unlike past years, where I erred about evenly in both directions, this year I was about 4% underconfident across the board. I&#8217;m not sure how much I should adjust and become more confident. In past years I&#8217;ve been burned by major black swan events that affect multiple predictions and made me look overconfident. In 2019 I tried to leave a cushion for that, but nothing too unexpected happened and I ended up playing it too safe. My worst failures were underestimating Bitcoin (but who didn&#8217;t?) and overestimating SpaceX&#8217;s ability to launch their crew on schedule. I didn&#8217;t check formally, but there doesn&#8217;t seem to be much difference in my calibration about world affairs vs. my personal life. </p>
<p>I forgot to make predictions for 2020 until now, which in retrospect was the best prediction I&#8217;ve ever made. I&#8217;ll probably come up with some later this month.</p>}
\end{xmlentriescontent}
\xmlentrieswfwcommentrss{https://slatestarcodex.com/2020/04/08/2019-predictions-calibration-results/feed/}\xmlentriesslashcomments{84}\xmlentriestitle{Never Tell Me The Odds (Ratio)}
\begin{xmlentriestitledetail}
\xmltitledetailtype{text/plain}\xmltitledetailbase{https://slatestarcodex.com/feed/}\xmltitledetailvalue{Never Tell Me The Odds (Ratio)}
\end{xmlentriestitledetail}

\begin{xmlentrieslinks}
\xmllinksrel{alternate}\xmllinkstype{text/html}\xmllinkshref{https://slatestarcodex.com/2020/04/07/never-tell-me-the-odds-ratio/}
\end{xmlentrieslinks}
\xmlentrieslink{https://slatestarcodex.com/2020/04/07/never-tell-me-the-odds-ratio/}\xmlentriescomments{https://slatestarcodex.com/2020/04/07/never-tell-me-the-odds-ratio/#comments}\xmlentriespublished{Wed, 08 Apr 2020 06:07:20 +0000}
\begin{xmlentriesauthors}
\xmlauthorsname{Scott Alexander}
\end{xmlentriesauthors}
\xmlentriesauthor{Scott Alexander}
\begin{xmlentriesauthordetail}
\xmlauthordetailname{Scott Alexander}
\end{xmlentriesauthordetail}

\begin{xmlentriestags}
\xmltagsterm{Uncategorized}\xmltagsterm{statistics}
\end{xmlentriestags}
\xmlentriesid{https://slatestarcodex.com/?p=5927}\xmlentriessummary{[Epistemic status: low confidence, someone tell me if the math is off. Title was stolen from an old Less Wrong post that seems to have disappeared &#8211; let me know if it&#8217;s yours and I&#8217;ll give you credit] I almost &#8230; <a href="https://slatestarcodex.com/2020/04/07/never-tell-me-the-odds-ratio/">Continue reading <span class="pjgm-metanav">&#8594;</span></a>}
\begin{xmlentriessummarydetail}
\xmlsummarydetailtype{text/html}\xmlsummarydetailbase{https://slatestarcodex.com/feed/}\xmlsummarydetailvalue{[Epistemic status: low confidence, someone tell me if the math is off. Title was stolen from an old Less Wrong post that seems to have disappeared &#8211; let me know if it&#8217;s yours and I&#8217;ll give you credit] I almost &#8230; <a href="https://slatestarcodex.com/2020/04/07/never-tell-me-the-odds-ratio/">Continue reading <span class="pjgm-metanav">&#8594;</span></a>}
\end{xmlentriessummarydetail}

\begin{xmlentriescontent}
\xmlcontenttype{text/html}\xmlcontentbase{https://slatestarcodex.com/feed/}\xmlcontentvalue{<p><font size="1"><i>[Epistemic status: low confidence, someone tell me if the math is off. Title was stolen from an old Less Wrong post that seems to have disappeared &#8211; let me know if it&#8217;s yours and I&#8217;ll give you credit]</i></font></p>
<p>I almost screwed up yesterday&#8217;s journal club. The study reported an odds ratio of 2.9 for antidepressants. Even though I <i>knew</i> <A HREF="http://itre.cis.upenn.edu/~myl/languagelog/archives/004767.html">odds ratios are terrible</A> and you should never trust your intuitive impression of them, I <i>still</i> mentally filed this away as &#8220;sounds like a really big effect&#8221;.</p>
<p>This time I was saved by Chen&#8217;s <A HREF="https://sci-hub.tw/10.1080/03610911003650383">How Big is a Big Odds Ratio? Interpreting the Magnitudes of Odds Ratios in Epidemiological Studies</A>, which explains how to convert ORs into effect sizes. Colored highlights are mine. I have followed the usual statistical practice of interpreting effect sizes of 0.2 as &#8220;small&#8221;, of 0.5 as &#8220;moderate&#8221;, and 0.8 as &#8220;large&#8221;, but feeling guilty about it.</p>
<p><center><IMG SRC="http://slatestarcodex.com/blog_images/chentable.png"></center></p>
<p>Based on <A HREF="https://stats.stackexchange.com/questions/352586/convert-odds-ratio-to-cohens-d-taking-rate-of-prevalence-into-account">this page</A>, I gather Chen has used some unusually precise formula to calculate this, but that a quick heuristic is to ignore the prevalence and just take [ln(odds ratio)]/1.81. </p>
<p>Suppose you run a drug trial. In your control group of 1000 patients, 300 get better on their own. In your experimental group of 1000 patients, 600 get better total (presumably 300 on their own, 300 because your drug worked). The <A HREF="https://www.medcalc.org/calc/relative_risk.php">relative risk calculator</A> says your relative risk of recovery on the drug is 2.0. <A HREF="https://www.theanalysisfactor.com/the-difference-between-relative-risk-and-odds-ratios/">Odds ratio</A> is 3.5, effect size is 0.7. So you&#8217;ve managed to double the recovery rate &#8211; in fact, to save an entire extra 30% of your population &#8211; and you <i>still</i> haven&#8217;t qualified for a &#8220;large&#8221; effect size.</p>
<p>The moral of the story is that (to me) odds ratios sound bigger than they really are, and effect sizes sound smaller, so you should be really careful comparing two studies that report their results differently.</p>}
\end{xmlentriescontent}
\xmlentrieswfwcommentrss{https://slatestarcodex.com/2020/04/07/never-tell-me-the-odds-ratio/feed/}\xmlentriesslashcomments{18}\xmlentriestitle{SSCJC: Real World Depression Measurement}
\begin{xmlentriestitledetail}
\xmltitledetailtype{text/plain}\xmltitledetailbase{https://slatestarcodex.com/feed/}\xmltitledetailvalue{SSCJC: Real World Depression Measurement}
\end{xmlentriestitledetail}

\begin{xmlentrieslinks}
\xmllinksrel{alternate}\xmllinkstype{text/html}\xmllinkshref{https://slatestarcodex.com/2020/04/06/sscjc-real-world-depression-measurement/}
\end{xmlentrieslinks}
\xmlentrieslink{https://slatestarcodex.com/2020/04/06/sscjc-real-world-depression-measurement/}\xmlentriescomments{https://slatestarcodex.com/2020/04/06/sscjc-real-world-depression-measurement/#comments}\xmlentriespublished{Tue, 07 Apr 2020 02:16:24 +0000}
\begin{xmlentriesauthors}
\xmlauthorsname{Scott Alexander}
\end{xmlentriesauthors}
\xmlentriesauthor{Scott Alexander}
\begin{xmlentriesauthordetail}
\xmlauthordetailname{Scott Alexander}
\end{xmlentriesauthordetail}

\begin{xmlentriestags}
\xmltagsterm{Uncategorized}\xmltagsterm{journal club}\xmltagsterm{psychiatry}
\end{xmlentriestags}
\xmlentriesid{https://slatestarcodex.com/?p=5925}\xmlentriessummary{The largest non-pharma antidepressant trial ever conducted just confirmed what we already knew: scientists love naming things after pandas. We already had PANDAS (Pediatric Autoimmune Neuropsychiatric Disorders Associated with Streptococcus) and PANDA (Proton ANnhilator At DArmstadt). But the latest in &#8230; <a href="https://slatestarcodex.com/2020/04/06/sscjc-real-world-depression-measurement/">Continue reading <span class="pjgm-metanav">&#8594;</span></a>}
\begin{xmlentriessummarydetail}
\xmlsummarydetailtype{text/html}\xmlsummarydetailbase{https://slatestarcodex.com/feed/}\xmlsummarydetailvalue{The largest non-pharma antidepressant trial ever conducted just confirmed what we already knew: scientists love naming things after pandas. We already had PANDAS (Pediatric Autoimmune Neuropsychiatric Disorders Associated with Streptococcus) and PANDA (Proton ANnhilator At DArmstadt). But the latest in &#8230; <a href="https://slatestarcodex.com/2020/04/06/sscjc-real-world-depression-measurement/">Continue reading <span class="pjgm-metanav">&#8594;</span></a>}
\end{xmlentriessummarydetail}

\begin{xmlentriescontent}
\xmlcontenttype{text/html}\xmlcontentbase{https://slatestarcodex.com/feed/}\xmlcontentvalue{<p>The largest non-pharma antidepressant trial ever conducted just confirmed what we already knew: scientists love naming things after pandas.</p>
<p>We already had <A HREF="https://en.wikipedia.org/wiki/PANDAS">PANDAS</A> (Pediatric Autoimmune Neuropsychiatric Disorders Associated with Streptococcus) and <A HREF="https://panda.gsi.de/">PANDA</A> (Proton ANnhilator At DArmstadt). But the latest in this pandemic of panda pandering is the <A HREF="https://www.thelancet.com/journals/lanpsy/article/PIIS2215-0366(19)30366-9/fulltext#figures">PANDA</A> (Prescribing ANtiDepressants Appropriately) Study. A group of British scientists followed 655 complicated patients who received either placebo or the antidepressant sertraline (Zoloft®).</p>
<p>The PANDA trial was unique in two ways. First, as mentioned, it was the largest ever trial for a single antidepressant not funded by a pharmaceutical company. Second, it was designed to mimic &#8220;the real world&#8221; as closely as possible. In most antidepressant trials, researchers wait to gather the perfect patients: people who definitely have depression and definitely don&#8217;t have anything else. Then they get top psychiatrists to carefully evaluate each patient, monitor the way they take the medication, and exhaustively test every aspect of their progress with complicated questionnaires. PANDA looked for normal people going to their GP&#8217;s (US English: PCP&#8217;s) office, with all of the mishmash of problems and comorbidities that implies.</p>
<p>Measuring real-world efficacy is especially important for antidepressant research because past studies have failed to match up with common sense. Most studies show antidepressants having &#8220;clinically insignificant&#8221; effects on depression; that is, although scientists can find a statistical difference between treatment and placebo groups, it seems too small to matter. But in the real world, most doctors find antidepressants very useful, and many patients credit them for impressive recoveries. Maybe a big real-world study would help bridge the gap between study vs. real-world results.</p>
<p>The study used an interesting selection criteria &#8211; you were allowed in if you and your doctor reported &#8220;uncertainty&#8230;about the possible benefit of an antidepressant&#8221;. That is, people who definitely didn&#8217;t need antidepressants were sent home without an antidepressant, people who definitely <i>did</i> need antidepressants got the antidepressant, and people on the borderline made it into the study. This is very different from the usual pharma company method of using the people who desperately need antidepressants the most in order to inflate success rates. And it&#8217;s more relevant to clinical practice &#8211; part of what it means for studies to guide our clinical practice is to tell us what to do in cases where we&#8217;re otherwise not sure. And unlike most studies, which use strict diagnostic criteria, this study just used a perception of needing help &#8211; not even necessarily for depression, some of these patients were anxious or had other issues. Again, more relevant for clinical practice, where the borders between depression, anxiety, etc aren&#8217;t always that clear.</p>
<p>They ended up with 655 people, ages 18-74, from Bristol, Liverpool, London, and York. They followed up on how they were doing at 2, 6, and 12 weeks after they started medication. As usual, they scored patients on a bunch of different psychiatric tests.</p>
<p>In the end, PANDA confirmed what we already know: it is really hard to measure antidepressant outcomes, and all the endpoints conflict with each other.</p>
<p>I am going to be <i>much</i> nicer to you than the authors of the original paper were to their readers, and give you a convenient table with all of the results converted to effect sizes. All values are positive, meaning the antidepressant group beat the placebo group. I calculated some of this by hand, so it may be wrong.</p>
<div class="table-responsive"><table  style="width:100%; "  class="easy-table easy-table-default " >
<tbody>
<tr><td ><b>Endpoint</b></td>
<td ><b>Effect size</b></td>
<td ><b>p-value</b></td>
</tr>

<tr><td >PHQ-9</td>
<td >0.19</td>
<td >0.1</td>
</tr>

<tr><td >BDI</td>
<td >0.21</td>
<td >0.01</td>
</tr>

<tr><td >GAD-7</td>
<td >0.25</td>
<td >≤0.0001</td>
</tr>

<tr><td >SF-12</td>
<td >0.23</td>
<td >0.0002</td>
</tr>

<tr><td >PHQ-9 Remission</td>
<td >0.31</td>
<td >0.1</td>
</tr>

<tr><td >BDI Remission</td>
<td >0.55</td>
<td >0.049</td>
</tr>

<tr><td >General improvement</td>
<td >0.49</td>
<td >≤0.0001</td>
</tr>
</tbody></table></div>
<p>PHQ-9 is a common depression test. BDI is another common depression test. GAD-7 is an anxiety test. SF-12 is a vague test of how mentally healthy you&#8217;re feeling. Remission indicates percent of patients whose test scores have improved enough that they qualify as &#8220;no longer depressed&#8221;. General improvement was just asking patients if they felt any better.</p>
<p>I like this study because it examines some of the mystery of why antidepressants do much worse in clinical trials than according to anecdotal doctor and patient intuitions. One possibility has always been that we&#8217;re measuring these things wrong. This study goes to exactly the kind of naturalistic setting where people report good results, and measures things a bunch of different ways to see what happens.</p>
<p>The results are broadly consistent with previous studies. Usually people think of effect sizes less than 0.2 as miniscule, less than 0.5 as small, and less than 0.8 as medium. This study showed only small to low-medium effect sizes for everything.</p>
<p>I haven&#8217;t checked whether differences between effect sizes were significant. But just eyeballing them, this study doesn&#8217;t agree with my hypothesis that SSRIs are better for anxiety than for depression; the GAD-7 effect size is about the same as the PHQ and BDI effect sizes.</p>
<p>It does weakly support a hypothesis where SSRIs are better for patient-rated improvement than for researcher-measured tests. The highest effect size was in &#8220;self-rated improvement&#8221;, where the researchers just asked the patients if they felt better. This effect size (0.49) was still small. But if we let ourselves round it up, it reaches all the way to &#8220;medium&#8221;. Progress!</p>
<p>What does this mean in real life? 59% of patients in the antidepressant group, compared to 42% of patients in the placebo group, said they felt better. I&#8217;m actually okay with this. It means that for every 58 patients who wouldn&#8217;t have gotten better on placebo, 17 of them would get better on an antidepressant &#8211; in other words, the antidepressant successfully converted 30% of people from nonresponder to responder. This obviously isn&#8217;t as good as 50% or 100%. But it doesn&#8217;t strike me as consistent with the claims of &#8220;clinically insignificant&#8221; and &#8220;why would anyone ever use these medications&#8221;?</p>
<p>(though of course, this is just one study, and it&#8217;s a study where I took the most promising of many different endpoints, so it&#8217;s not exactly cause for celebration)</p>
<p>If antidepressants do better on patient report than on our depression tests, does that mean our depression tests are bad? Maybe. Figure 4 from <A HREF="https://sci-hub.tw/10.1016/s2215-0366(19)30216-0">Hieronymous et al</a> helps clarify a bit of what&#8217;s going on:</p>
<p><center><IMG SRC="http://slatestarcodex.com/blog_images/hieronymous.png"></center></p>
<p>At least in less severely depressed patients, antidepressants are more likely to produce significant gains on vaguer or more fundamental symptoms (like &#8220;depressed mood&#8221; or &#8220;anxiety&#8221;) than on specific symptoms (like insomnia or psychomotor disruptions). Probably patients care a lot less about &#8220;psychomotor disruptions&#8221; than researchers studying depression do, and they just want to feel happy again. This study&#8217;s finding of an 0.4 &#8211; 0.5 effect size on patient response closely matches Hieronymous et al&#8217;s finding of an 0.4 &#8211; 0.5 effect size on depressed mood.</p>
<p>Like most studies, PANDA used a one-size-fits-all solution based on a single antidepressant. This is a reasonable choice for a study, but doesn&#8217;t match clinical practice, where we usually try one antidepressant, see if it works, and try another if it doesn&#8217;t. In patients like the ones in the study, who had failed treatment with sertraline, a usual next step would be to try bupropion. An even better idea would be to screen patients for more typical vs. atypical depression, start people on sertraline or bupropion based on the symptom profile, and then switch to the other if the first one didn&#8217;t work. The STAR*D trial did something like this, and got better results than an SSRI alone. I haven&#8217;t done the work I would need to compare this to STAR*D, but it seems possible that the extra push from targeted treatment could bring our 0.49 effect size up to the 0.7 or 0.8 level where we could actually feel fully confident prescribing this stuff.</p>}
\end{xmlentriescontent}
\xmlentrieswfwcommentrss{https://slatestarcodex.com/2020/04/06/sscjc-real-world-depression-measurement/feed/}\xmlentriesslashcomments{61}\xmlentriestitle{Open Thread 151}
\begin{xmlentriestitledetail}
\xmltitledetailtype{text/plain}\xmltitledetailbase{https://slatestarcodex.com/feed/}\xmltitledetailvalue{Open Thread 151}
\end{xmlentriestitledetail}

\begin{xmlentrieslinks}
\xmllinksrel{alternate}\xmllinkstype{text/html}\xmllinkshref{https://slatestarcodex.com/2020/04/05/open-thread-151/}
\end{xmlentrieslinks}
\xmlentrieslink{https://slatestarcodex.com/2020/04/05/open-thread-151/}\xmlentriescomments{https://slatestarcodex.com/2020/04/05/open-thread-151/#comments}\xmlentriespublished{Sun, 05 Apr 2020 21:20:19 +0000}
\begin{xmlentriesauthors}
\xmlauthorsname{Scott Alexander}
\end{xmlentriesauthors}
\xmlentriesauthor{Scott Alexander}
\begin{xmlentriesauthordetail}
\xmlauthordetailname{Scott Alexander}
\end{xmlentriesauthordetail}

\begin{xmlentriestags}
\xmltagsterm{Uncategorized}\xmltagsterm{open}
\end{xmlentriestags}
\xmlentriesid{https://slatestarcodex.com/?p=5923}\xmlentriessummary{This is the bi-weekly visible open thread (there are also hidden open threads twice a week you can reach through the Open Thread tab on the top of the page). Post about anything you want, but please try to avoid &#8230; <a href="https://slatestarcodex.com/2020/04/05/open-thread-151/">Continue reading <span class="pjgm-metanav">&#8594;</span></a>}
\begin{xmlentriessummarydetail}
\xmlsummarydetailtype{text/html}\xmlsummarydetailbase{https://slatestarcodex.com/feed/}\xmlsummarydetailvalue{This is the bi-weekly visible open thread (there are also hidden open threads twice a week you can reach through the Open Thread tab on the top of the page). Post about anything you want, but please try to avoid &#8230; <a href="https://slatestarcodex.com/2020/04/05/open-thread-151/">Continue reading <span class="pjgm-metanav">&#8594;</span></a>}
\end{xmlentriessummarydetail}

\begin{xmlentriescontent}
\xmlcontenttype{text/html}\xmlcontentbase{https://slatestarcodex.com/feed/}\xmlcontentvalue{<p>This is the bi-weekly visible open thread (there are also hidden open threads twice a week you can reach through the Open Thread tab on the top of the page). Post about anything you want, but please try to avoid hot-button political and social topics. You can also talk at the <b><A HREF="https://www.reddit.com/r/slatestarcodex/">SSC subreddit</A></b> &#8211; and also check out the <b><A HREF="http://sscpodcast.libsyn.com/rss">SSC Podcast</A></b>.  Also:</p>
<p><b>1.</b> Comment of the week: <A HREF="https://slatestarcodex.com/2020/04/01/book-review-the-precipice/#comment-874316">Bean questions</A> <i>The Precipice</i>&#8216;s assessment of nuclear winter, links back to a <A HREF="https://slatestarcodex.com/2016/04/11/ot47-openai/#comment-344892">comment thread from years ago</A> challenging the Robock paper Ord relies on.</p>
<p><b>2.</b> New sidebar ad: this one is for SafetyWing, which bills itself as &#8220;Norwegian founders with an international team on a mission to offer the equivalent of a Norwegian social safety net globally available as a membership &#8211; currently offering <A HREF="https://safetywing.com/nomad-insurance">travel medical insurance for nomads</A>, and <A HREF="https://safetywing.com/remote-health">global health insurance for remote teams</A>.&#8221; It also has a Medium article where it claims its end goal is to <A HREF="https://medium.com/@safetywingcom/a-practical-guide-to-building-a-country-on-the-internet-691ac9da3b0e">build a country on the Internet</A>, which is a delightfully tech-startup-Medium-article thing to claim.</p>
<p><b>3.</b> As always, I apologize for being bad at answering emails. In some cases, I am weeks behind. In other cases, I am grateful for what you have to say but have given up on responding. In others, I have said &#8220;that&#8217;s interesting, I&#8217;ll check it out&#8221; while secretly knowing that I will never do that. Expect this kind of thing to continue.</p>
<p><b>4.</b> Dan Wahl has <A HREF="https://danwahl.net/unsong-audiobook/">an automated Unsong audiobook</A>. </p>
<p><b>5.</b> Calling researchers and lawyers &#8211; does anyone know the legalities of running an informal study without IRB approval? IE, if I were to email the SSC mailing list recommending people try a certain vitamin, nobody would think twice of it. If I were to email the SSC mailing list asking people with last names A-M to try the vitamin, and people with last names N-Z not to, and to record their results and send them to me, how much trouble would I be in? If the answer is &#8220;not much&#8221;, what does the requirement that &#8220;studies&#8221; have an IRB mean? Where do you cross the line from neat decentralized science experiment to official study, and what happens to people if they end up on the wrong side of it?</p>}
\end{xmlentriescontent}
\xmlentrieswfwcommentrss{https://slatestarcodex.com/2020/04/05/open-thread-151/feed/}\xmlentriesslashcomments{1116}\xmlentriestitle{Book Review: The Precipice}
\begin{xmlentriestitledetail}
\xmltitledetailtype{text/plain}\xmltitledetailbase{https://slatestarcodex.com/feed/}\xmltitledetailvalue{Book Review: The Precipice}
\end{xmlentriestitledetail}

\begin{xmlentrieslinks}
\xmllinksrel{alternate}\xmllinkstype{text/html}\xmllinkshref{https://slatestarcodex.com/2020/04/01/book-review-the-precipice/}
\end{xmlentrieslinks}
\xmlentrieslink{https://slatestarcodex.com/2020/04/01/book-review-the-precipice/}\xmlentriescomments{https://slatestarcodex.com/2020/04/01/book-review-the-precipice/#comments}\xmlentriespublished{Thu, 02 Apr 2020 05:40:43 +0000}
\begin{xmlentriesauthors}
\xmlauthorsname{Scott Alexander}
\end{xmlentriesauthors}
\xmlentriesauthor{Scott Alexander}
\begin{xmlentriesauthordetail}
\xmlauthordetailname{Scott Alexander}
\end{xmlentriesauthordetail}

\begin{xmlentriestags}
\xmltagsterm{Uncategorized}\xmltagsterm{book review}\xmltagsterm{rationality}
\end{xmlentriestags}
\xmlentriesid{https://slatestarcodex.com/?p=5920}\xmlentriessummary{I. It is a well known fact that the gods hate prophets. False prophets they punish only with ridicule. It&#8217;s the true prophets who have to watch out. The gods find some way to make their words come true in &#8230; <a href="https://slatestarcodex.com/2020/04/01/book-review-the-precipice/">Continue reading <span class="pjgm-metanav">&#8594;</span></a>}
\begin{xmlentriessummarydetail}
\xmlsummarydetailtype{text/html}\xmlsummarydetailbase{https://slatestarcodex.com/feed/}\xmlsummarydetailvalue{I. It is a well known fact that the gods hate prophets. False prophets they punish only with ridicule. It&#8217;s the true prophets who have to watch out. The gods find some way to make their words come true in &#8230; <a href="https://slatestarcodex.com/2020/04/01/book-review-the-precipice/">Continue reading <span class="pjgm-metanav">&#8594;</span></a>}
\end{xmlentriessummarydetail}

\begin{xmlentriescontent}
\xmlcontenttype{text/html}\xmlcontentbase{https://slatestarcodex.com/feed/}\xmlcontentvalue{<p><b>I.</b></p>
<p>It is a well known fact that the gods hate prophets. </p>
<p>False prophets they punish only with ridicule. It&#8217;s the true prophets who have to watch out. The gods find some way to make their words come true in the most ironic way possible, the one where knowing the future just makes things worse. The Oracle of Delphi told Croesus he would destroy a great empire, but when he rode out to battle, the empire he destroyed was his own. Zechariah predicted the Israelites would rebel against God; they did so by killing His prophet Zechariah. Jocasta heard a prediction that she would marry her infant son Oedipus, so she left him to die on a mountainside &#8211; ensuring neither of them recognized each other when he came of age.</p>
<p>Unfortunately for him, Oxford philosopher Toby Ord is a true prophet. He spent years writing his magnum opus <i><A HREF="https://www.amazon.com/Precipice-Existential-Risk-Future-Humanity-ebook/dp/B07V9GHKYP/ref=as_li_ss_tl?crid=YN9TIAZ7T8IC&#038;keywords=the+precipice&#038;qid=1585806254&#038;sprefix=the+precipi,aps,212&#038;sr=8-2&#038;&#038;linkCode=ll1&#038;tag=slatestarcode-20&#038;linkId=9ab3f571aa7f94cc9e36adbd9281f86b&#038;language=en_US">The Precipice</A></i>, warning that humankind was unprepared for various global disasters like pandemics and economic collapses. You can guess what happened next. His book came out March 3, 2020, in the middle of a global pandemic and economic collapse. He couldn&#8217;t go on tour to promote it, on account of the pandemic. Nobody was buying books anyway, on account of the economic collapse. All the newspapers and journals and so on that would usually cover an exciting new book were busy covering the pandemic and economic collapse instead. The score is still gods one zillion, prophets zero. So Ord&#8217;s PR person asked me to help spread the word, and here we are.</p>
<p>Imagine you were sent back in time to inhabit the body of Adam, primordial ancestor of mankind. It turns out the Garden of Eden has motorcycles, and Eve challenges you to a race. You know motorcycles can be dangerous, but you&#8217;re an adrenaline junkie, naturally unafraid of death. And it would help take your mind off that ever-so-tempting Tree of Knowledge. Do you go?</p>
<p>Before you do, consider that you&#8217;re not just risking your own life. A fatal injury to either of you would snuff out the entire future of humanity. Every song ever composed, every picture ever painted, every book ever written by all the greatest authors of the millennia would die stillborn. Millions of people would never meet their true loves, or get to raise their children. All of the triumphs and tragedies of humanity, from the conquests of Alexander to the moon landings, would come to nothing if you hit a rock and cracked your skull.</p>
<p>So maybe you shouldn&#8217;t motorcycle race. Maybe you shouldn&#8217;t even go outside. Maybe you and Eve should hide, panicked, in the safest-looking cave you can find.</p>
<p>Ord argues that 21st century humanity is in much the same situation as Adam. The potential future of the human race is vast. We have another five billion years until the sun goes out, and 10^100 until the universe becomes uninhabitable. Even with conservative assumptions, the galaxy could support quintillions of humans. Between Eden and today, the population would have multiplied five billion times; between today and our galactic future, it could easily multiply another five billion. However awed Adam and Eve would have been when they considered the sheer size of the future that depended on them, we should be equally awed.</p>
<p>So maybe we should do the equivalent of not motorcycling. And that would mean taking existential risks (&#8220;x-risks&#8221;) &#8211; disasters that might completely destroy humanity or permanently ruin its potential &#8211; very seriously. Even more seriously than we would take them just based on the fact that we don&#8217;t want to die. Maybe we should consider all counterbalancing considerations &#8211; &#8220;sure, global warming might be bad, but we also need to keep the economy strong!&#8221; &#8211; to be overwhelmed by the crushing weight of the future.</p>
<p>This is my metaphor, not Ord&#8217;s. He uses a different one &#8211; the Cuban Missile Crisis. We all remember the Cuban Missile Crisis as a week where humanity teetered on the precipice of destruction, then recovered into a more stable not-immediately-going-to-destroy-itself state. Ord speculates that far-future historians will remember the entire 1900s and 2000s as a sort of centuries-long Cuban Missile Crisis, a crunch time when the world was unusually vulnerable and everyone had to take exactly the right actions to make it through. Or as the namesake precipice, a <A HREF="http://www.slate.com/blogs/atlas_obscura/2013/10/24/a_terrifying_tour_of_the_world_s_most_dangerous_road_north_yungas_in_bolivia.html">place where</A> the road to the Glorious Future crosses a narrow rock ledge hanging over a deep abyss.</p>
<p>Ord has other metaphors too, and other arguments. The first sixty pages of <i>Precipice</i> are a series of arguments and thought experiments intended to drive home the idea that everyone dying would be really bad. Some of them were new to me and quite interesting &#8211; for example, an argument that we should keep the Earth safe for future generations as a way of &#8220;paying it forward&#8221; to our ancestors, who kept it safe for us. At times, all these arguments against allowing the destruction of the human race felt kind of excessive &#8211; isn&#8217;t there widespread agreement on this point? Even when there is disagreement, Ord doesn&#8217;t address it here, banishing counterarguments to various appendices &#8211; one arguing against time discounting the value of the future, another arguing against ethical theories that deem future lives irrelevant. This part of the book isn&#8217;t trying to get into the intellectual weeds. It&#8217;s just saying, again and again, that it would be <i>really bad</i> if we all died.</p>
<p>It&#8217;s tempting to psychoanalyze Ord here, with help from passages like this one:</p>
<blockquote><p>I have not always been focused on protecting our longterm future, coming to the topic only reluctantly. I am a philosopher at Oxford University, specialising in ethics. My earlier work was rooted in the more tangible concerns of global health and global poverty &#8211; in how we could best help the worst off. When coming to grips with these issues I felt the need to take my work in ethics beyond the ivory tower. I began advising the World Health Organization, World Bank, and UK government on the ethics of global health. And finding that my own money could do hundreds of times as much good for those in poverty as it could do for me, I made a lifelong pledge to donate at least a tenth of all I earn to help them. I founded a society, Giving What We Can, for those who wanted to join me, and was heartened to see thousands of people come together to pledge more than one billion pounds over our lifetimes to the most effective charities we know of, working on the most important causes. Together, we&#8217;ve already been able to transform the lives of thousands of people. And because there are many other ways beyond our donations in which we can help fashion a better world, I helped start a wider movement, known as &#8220;effective altruism&#8221;, in which people aspire to use prudence and reason to do as much good as possible.</p></blockquote>
<p>We&#8217;re in the Garden of Eden, so we should stop worrying about motorcycling and start worrying about protecting our future. But Ord&#8217;s equivalent of &#8220;motorcycling&#8221; was advising governments and NGOs on how best to fight global poverty. I&#8217;m familiar with his past work in this area, and he was amazing at it. He stopped because he decided that protecting the long-term future was more important. What must he think of the rest of us, who aren&#8217;t stopping our ordinary, non-saving-thousands-of-people-from-poverty day jobs?</p>
<p>In writing about Ord&#8217;s colleagues in the <A HREF="https://slatestarcodex.com/2017/08/16/fear-and-loathing-at-effective-altruism-global-2017/">effective altruist movement</A>, I quoted Larissa MacFarquahar on Derek Parfit:</p>
<blockquote><p>When I was interviewing him for the first time, for instance, we were in the middle of a conversation and suddenly he burst into tears. It was completely unexpected, because we were not talking about anything emotional or personal, as I would define those things. I was quite startled, and as he cried I sat there rewinding our conversation in my head, trying to figure out what had upset him. Later, I asked him about it. It turned out that what had made him cry was the idea of suffering. We had been talking about suffering in the abstract. I found that very striking.</p></blockquote>
<p>Toby Ord was Derek Parfit&#8217;s grad student, and I get sort of the same vibe from him &#8211; someone whose reason and emotions are unusually closely aligned. Stalin&#8217;s maxim that &#8220;one death is a tragedy, a million deaths is a statistic&#8221; accurately describes how most of us think. I am not sure it describes Toby Ord. I can&#8217;t say confidently that Toby Ord feels <i>exactly</i> a million times more intense emotions when he considers a million deaths than when he considers one death, but the scaling factor is definitely up there. When he considers ten billion deaths, or the deaths of the trillions of people who might inhabit our galactic future, he &#8211; well, he&#8217;s reduced to writing sixty pages of arguments and metaphors trying to cram into our heads exactly how bad this would be.</p>
<p><b>II.</b></p>
<p>The second part of the book is an analysis of specific risks, how concerned we should be about each, and what we can do to prevent them. Ord stays focused on <i>existential</i> risks here. He is not very interested in an asteroid that will wipe out half of earth&#8217;s population; the other half of humanity will survive to realize our potential. He&#8217;s not <i>completely</i> uninterested &#8211; wiping out half of earth&#8217;s population could cause some chaos that makes it harder to prepare for other catastrophes. But his main focus is on things that would kill everybody &#8211; or at least leave too few survivors to effectively repopulate the planet.</p>
<p>I expected Ord to be alarmist here. He is writing a book about existential risks, whose thesis is that we should take them extremely seriously. Any other human being alive would use this as an opportunity to play up how dangerous these risks are. Ord is too virtuous. Again and again, he knocks down bad arguments for worrying too much, points out that killing every single human being on earth, including the ones in Antarctic research stations, is actually quite hard, and ends up convincing me to downgrade my risk estimate.</p>
<p>So for example, we can rule out a high risk of destruction by any natural disaster &#8211; asteroid, supervolcano, etc &#8211; simply because these things haven&#8217;t happened before in our species&#8217; 100,000 year-odd history. Dino-killer sized asteroids seem to strike the Earth about once every few hundred million years, bounding the risk per century around the one-in-a-million level. But also, scientists are tracking almost all the large asteroids in the solar system, and when you account for their trajectories, the chance that one slips through and hits us in the next century goes down to less than one in a hundred fifty million. Large supervolcanoes seem to go off about once every 80,000 years, so the risk per century is 1/800. There are similar arguments around nearby supernovae, gamma ray bursts, and a bunch of other things.</p>
<p>I usually give any statistics I read a large penalty for &#8220;or maybe you&#8217;re a moron&#8221;. For example, lots of smart people said in 2016 that the chance of Trump winning was only 1%, or 0.1%, or 0.00001%, or whatever. But also, they were morons. They were using models, and their models were egregiously wrong. If you hear a person say that their model&#8217;s estimate of something is 0.00001%, very likely <i>your</i> estimate of the thing should be much higher than that, because maybe they&#8217;re a moron. I explain this in more detail <A HREF="https://www.lesswrong.com/posts/GrtbTAPfkJa4D6jjH/confidence-levels-inside-and-outside-an-argument">here</A>.</p>
<p>Ord is one of a tiny handful of people who doesn&#8217;t need this penalty. He explains this entire dynamic to his readers, agrees it is important, and adjusts several of his models appropriately. He is always careful to add a term for unknown unknowns &#8211; sometimes he is able to use clever methods to bound this term, other times he just takes his best guess. And he tries to use empirically-based methods that don&#8217;t have this problem, list his assumptions explicitly, and justify each assumption, so that you rarely have to rely on arguments shakier than &#8220;asteroids will continue to hit our planet at the same rate they did in the past&#8221;. I am really impressed with the care he puts into every argument in the book, and happy to accept his statistics at face value. People with no interest in x-risk may enjoy reading this book purely as an example of statistical reasoning done with beautiful lucidity.</p>
<p>When you accept very low numbers at face value, it can have strange implications. For example, should we study how to deflect asteroids? Ord isn&#8217;t sure. The base rate of asteroid strikes is so low that it&#8217;s outweighed by almost any change in the base rate. If we successfully learn how to deflect asteroids, that not only lets good guys deflect asteroids away from Earth, but also lets bad guys deflect asteroids <i>towards</i> Earth. The chance that an dino-killer asteroid approaches Earth and needs to be deflected away is 1/150 million per century, with small error bars. The chance that malicious actors deflect an asteroid towards Earth is much harder to figure out, but it has wide error bars, and there are a <i>lot</i> of numbers higher than 1/150 million. So probably most of our worry about asteroids over the next century should involve somebody using one as a weapon, and studying asteroid deflection probably makes that worse and not better.</p>
<p>Ord uses similar arguments again and again. Humanity has survived 100,000 years, so its chance of death by natural disaster per century is probably less than 1 / 1,000 (for complicated statistical reasons, he puts it at between 1/10,000 and 1/100,000). But humanity has only had technology (eg nuclear weapons, genetically engineered bioweapons) for a few decades, so there are no such guarantees of its safety. Ord thinks the overwhelming majority of existential risk comes from this source, and singles out four particular technological risks as most concerning.</p>
<p><u>First</u>, nuclear war. This was one of the places where Ord&#8217;s work is cause for optimism. You&#8217;ve probably heard that there are enough nuclear weapons to &#8220;destroy the world ten times over&#8221; or something like that. There aren&#8217;t. There are enough nuclear weapons to destroy lots of majors city, kill the majority of people, and cause a very bad nuclear winter for the rest. But there aren&#8217;t enough to literally kill every single human being. And because of the way the Earth&#8217;s climate works, the negative effects of nuclear winter would probably be concentrated in temperate and inland regions. Tropical islands and a few other distant locales (Ord suggests Australia and New Zealand) would have a good chance of making it through even a large nuclear apocalypse with enough survivors to repopulate the Earth. A lot of things would have to go wrong at once, and a lot of models be flawed in ways they don&#8217;t seem to be flawed, for a nuclear war to literally kill everyone. Ord gives the per-century risk of extinction from this cause at 1 in 1,000.</p>
<p><u>Second</u>, global warming. The <A HREF="https://www.vox.com/future-perfect/2019/6/13/18660548/climate-change-human-civilization-existential-risk">current scientific consensus</A> is that global warming will be really bad but not literally kill every single human. Even for implausibly high amounts of global warming, survivors can always flee to a pleasantly balmy Greenland. The main concern from an x-risk point of view is &#8220;runaway global warming&#8221; based on strong feedback loops. For example, global warming causes permafrost to melt, which releases previously trapped carbon, causing more global warming, causing more permafrost to melt, etc. Or global warming causes the oceans to warm, which makes them release more methane, which causes more global warming, causing the oceans to get warmer, etc. In theory, this could get really bad &#8211; something similar seems to have happened on Venus, which now has an average temperature of 900 degrees Fahrenheit. But Ord thinks it probably won&#8217;t happen here. His worst-case scenario estimates 13 &#8211; 20 degrees C of warming by 2300. This is really bad &#8211; summer temperatures in San Francisco would occasionally pass 140F &#8211; but still well short of Venus, and compatible with the move-to-Greenland scenario. Also, global temperatures jumped 5 degree C (to 14 degrees above current levels) fifty million years ago, and this didn&#8217;t seem to cause Venus-style runaway warming. This isn&#8217;t a perfect analogy for the current situation, since the current temperature increase is happening faster than the ancient one did, but it&#8217;s still a reason for hope. This is another one that could easily be an apocalyptic tragedy unparalleled in human history but probably not an existential risk; Ord estimates the x-risk per century as 1/1,000.</p>
<p>The same is true for other environmental disasters, of which Ord discusses a long list. Overpopulation? No, fertility rates have crashed and the population is barely expanding anymore (also, it&#8217;s hard for overpopulation to cause human extinction). Resource depletion? New discovery seems to be faster than depletion for now, and society could route around most plausible resources shortages. Honeybee collapse? Despite what you&#8217;ve heard, losing all pollinators would only cause a 3 &#8211; 8% decrease in global crop production. He gives all of these combined plus environmental unknown unknowns an additional 1/1,000, just in case.</p>
<p><u>Third</u>, pandemics. Even though pathogens are natural, Ord classifies pandemics as technological disasters for two reasons. First, natural pandemics are probably getting worse because our technology is making cities denser, linking countries closer together, and bringing humans into more contact with the animal vectors of zoonotic disease (in one of the book&#8217;s more prophetic passages, Ord mentions the risk of a disease crossing from bats to humans). But second, bioengineered pandemics may be especially bad. These could be either accidental (surprisingly many biologists alter diseases to make them worse as part of apparently legitimate scientific research) or deliberate (bioweapons). There are enough unknown unknowns here that Ord is uncomfortable assigning relatively precise and low risk levels like he did in earlier categories, and this section generally feels kind of rushed, but he estimates the per-century x-risk from natural pandemics as 1/10,000 and from engineered pandemics as 1/30. </p>
<p>The <u>fourth</u> major technological risk is AI. You&#8217;ve all <A HREF="https://slatestarcodex.com/2020/01/30/book-review-human-compatible/">read about</A> <A HREF="https://slatestarcodex.com/2019/08/27/book-review-reframing-superintelligence/">this one</A> by now, so I won&#8217;t go into the details, but it fits the profile of a genuinely dangerous risk. It&#8217;s related to technological advance, so our long and illustrious history of not dying from it thus far offers no consolation. And because it could be actively trying to eliminate humanity, isolated populations on islands or in Antarctica or wherever offer less consolation than usual. Using the same arguments and <A HREF="https://slatestarcodex.com/2017/06/08/ssc-journal-club-ai-timelines/">sources</A> we&#8217;ve seen every other time this topic gets brought up, Ord assigns this a 1/10 risk per century, the highest of any of the scenarios he examines, writing:</p>
<blockquote><p>In my view, the greatest risk to humanity’s potential in the next hundred years comes from unaligned artificial intelligence, which I put at 1 in 10. One might be surprised to see such a high number for such a speculative risk, so it warrants some explanation.</p>
<p>A common approach to estimating the chance of an unprecedented event with earth-shaking consequences is to take a sceptical stance: to start with an extremely small probability and only raise it from there when a large amount of hard evidence is presented. But I disagree. Instead, I think that the right method is to start with a probability that reflects our overall impressions, then adjust this in light of the scientific evidence. When there is a lot of evidence, these approaches converge. But when there isn’t, the starting point can matter.</p>
<p>In the case of artificial intelligence, everyone agrees the evidence and arguments are far from watertight, but the question is where does this leave us? Very roughly, my approach is to start with the overall view of the expert community that there is something like a 1 in 2 chance that AI agents capable of outperforming humans in almost every task will be developed in the coming century. And conditional on that happening, we shouldn’t be shocked if these agents that outperform us across the board were to inherit our future.</p></blockquote>
<p>The book also addresses a few more complicated situations. There are ways humankind could fail to realize its potential even without being destroyed. For example, if it got trapped in some kind of dystopia that it was impossible to escape. Or if it lost so many of its values that we no longer recognized it as human. Ord doesn&#8217;t have too much to say about these situations besides acknowledging that they would be bad and need further research. Or a series of disasters could each precipitate one another, or a minor disaster could leave people unprepared for a major disaster, or something along those lines.</p>
<p>Here, too, Ord is more optimistic than some other sources I have read. For example, some people say that if civilization ever collapses, it will never be able to rebuild, because we&#8217;ve already used up all easily-accessible sources of eg fossil fuels, and an infant civilization won&#8217;t be able to skip straight from waterwheels to nuclear. Ord is more sanguine:</p>
<blockquote><p>Even if civilization did collapse, it is likely that it could be re-established. As we have seen, civilization has already been independently established at least seven times by isolated peoples. While one might think resources depletion could make this harder, it is more likely that it has become substantially easier. Most dissasters short of human extinction would leave our domesticated animals and plants, as well as copious material resources in the ruins of our cities &#8211; it is much easier to re-forge iron from old railings than to smelt it from ore. Even expendable resources such as coal would be much easier to access, via abandoned reserves and mines, than they ever were in the eighteenth century. Moreover, evidence that civilisation is possible, and the tools and knowledge to help rebuild, would be scattered across the world.</p></blockquote>
<p><b>III.</b></p>
<p>Still, these risks are real, and humanity will under-respond to them for predictable reasons. </p>
<p>First, free-rider problems. If some people invest resources into fighting these risks and others don&#8217;t, both sets of people will benefit equally. So all else being equal everyone would prefer that someone else do it. We&#8217;ve already seen this play out with international treaties on climate change.</p>
<p>Second, scope insensitivity. A million deaths, a billion deaths, and complete destruction of humanity all sound like such unimaginable catastrophes that they&#8217;re hardly worth differentiating. But plausibly we should put 1000x more resources into preventing a billion deaths than a million, and some further very large scaling factor into preventing human extinction. People probably won&#8217;t think that way, which will further degrade our existential risk readiness.</p>
<p>Third, availability bias. Existential risks have never happened before. Even their weaker non-omnicidal counterparts have mostly faded into legend &#8211; the Black Death, the Tunguska Event. The current pandemic is a perfect example. Big pandemics happen once every few decades &#8211; the Spanish flu of 1918 and the Hong Kong Flu of 1968 are the most salient recent examples. Most countries put some effort into preparing for the next one. But the preparation wasn&#8217;t very impressive. After this year, I bet we&#8217;ll put impressive effort into preparing for respiratory pandemics the next decade or two, while continuing to ignore other risks like solar flares or megadroughts that are equally predictable. People feel weird putting a lot of energy into preparing for something that has never happened before, and their value of &#8220;never&#8221; is usually &#8220;in a generation or two&#8221;. Getting them to care about things that have <i>literally</i> never happened before, like climate change, nuclear winter, or AI risk, is an even taller order.</p>
<p>And even when people seem to care about distant risks, it can feel like a half-hearted effort. During a Berkeley meeting of the Manhattan Project, Edward Teller brought up the basic idea behind the hydrogen bomb. You would use a nuclear bomb to ignite a self-sustaining fusion reaction in some other substance, which would produce a bigger explosion than the nuke itself. The scientists got to work figuring out what substances could support such reactions, and found that they couldn&#8217;t rule out nitrogen-14. The air is 79% nitrogen-14. If a nuclear bomb produced nitrogen-14 fusion, it would ignite the atmosphere and turn the Earth into a miniature sun, killing everyone. They hurriedly convened a task force to work on the problem, and it reported back that neither nitrogen-14 nor a second candidate isotope, lithium-7, could support a self-sustaining fusion reaction.</p>
<p>They seem to have been <i>moderately</i> confident in these calculations. But there was enough uncertainty that, when the Trinity test produced a brighter fireball than expected, Manhattan Project administrator James Conant was &#8220;overcome with dread&#8221;, believing that atmospheric ignition had happened after all and the Earth had only seconds left. And later, the US detonated a bomb whose fuel was contaminated with lithium-7, the explosion was much bigger than expected, and some bystanders were killed. It turned out atomic bombs could initiate lithium-7 fusion after all! As Ord puts it, &#8220;of the two major thermonuclear calculations made that summer at Berkeley, they got one right and one wrong&#8221;. This doesn&#8217;t really seem like the kind of crazy anecdote you could tell in a civilization that was taking existential risk seriously enough.</p>
<p>So what should we do? That depends who you mean by &#8220;we&#8221;.</p>
<p>Ordinary people should follow the standard advice of effective altruism. If they feel like their talents are suited for a career in this area, they should check out <A HREF="https://80000hours.org/">80,000 Hours</A> and similar resources and try to pursue it. Relevant careers include science (developing helpful technologies to eg capture carbon or understand AI), politics and policy (helping push countries to take risk-minimizing actions), and general thinkers and influencers (philosophers to remind us of our ethical duties, journalists to help keep important issues fresh in people&#8217;s minds). But also, anything else that generally strengthens and stabilizes the world. Diplomats who help bring countries closer together, since international peace reduces the risk of nuclear war and bioweapons and makes cooperation against other threats more likely. Economists who help keep the market stable, since a prosperous country is more likely to have resources to devote to the future. Even teachers are helping train the next generation of people who can help in the effort, although Ord warns against going <i>too</i> meta &#8211; most people willing to help with this will still be best off working on causes that affect existential risk directly. If they don&#8217;t feel like their talents lie in any of these areas, they can keep earning money at ordinary jobs and donate some of it (traditionally 10%) to x-risk related charities.</p>
<p>Rich people, charitable foundations, and governments should fund anti-x-risk work more than they&#8217;re already doing. Did you know that the Biological Weapons Convention, a key international agreement banning biological warfare, has a budget lower than the average McDonald&#8217;s restaurant (not total McDonald corporate profits, a single restaurant)? Or that total world spending on preventing x-risk is less than total world spending on ice cream? Ord suggests a target of between 0.1% and 1% of gross world product for anti-x-risk efforts. </p>
<p>And finally, Ord has a laundry list of requests for sympathetic policy-makers (Appendix F). Most of them are to put more research and funding into things, but the actionable specific ones are: restart various nuclear disarmament treaties, take ICBMs off &#8220;hair-trigger alert&#8221;, have the US rejoin the Paris Agreement on climate change, fund the Biological Weapons Convention better, and mandate that DNA synthesis companies screen consumer requests for dangerous sequences so that terrorists can&#8217;t order a batch of smallpox virus (80% of companies currently do this screening, but 20% don&#8217;t). The actual appendix is six pages long, there are a <i>lot</i> of requests to put more research and funding into things.</p>
<p>In the last section, Ord explains that all of this is just the first step. After we&#8217;ve conquered existential risk (and all our other problems), we&#8217;ll have another task: to contemplate how we want to guide the future. Before we spread out into the galaxy, we might want to take a few centuries to sit back and think about what our obligations are to each other, the universe, and the trillions of people who may one day exist. We cannot take infinite time for this; the universe is expanding, and for each year we spend not doing interstellar colonization, three galaxies cross the <A HREF="https://en.wikipedia.org/wiki/Cosmological_horizon#Event_horizon">cosmological event horizon</A> and become forever unreachable, and all the potential human civilizations that might have flourished there come to nothing. Ord expects us to be concerned about this, and tries to reassure us that it will be okay (the <i>relative</i> loss each year is only one five-billionth of the universe). So he thinks taking a few centuries to reflect before beginning our interstellar colonization is worthwhile on net. But for now, he thinks this process should take a back seat to safeguarding the world from x-risk. Deal with the Cuban Missile Crisis we&#8217;re perpetually in the middle of, and <i>then</i> we&#8217;ll have time for normal philosophy.</p>
<p><b>IV.</b></p>
<p>In the spirit of highly-uncertain-estimates <A HREF="https://slatestarcodex.com/2013/05/02/if-its-worth-doing-its-worth-doing-with-made-up-statistics/">being better than no estimates at all</A>, Ord offers this as a draft of where the existential risk community is right now (&#8220;they are not in any way the final word, but are a concise summary of all I know about the risk landscape&#8221;):</p>
<p><center><IMG SRC="http://slatestarcodex.com/blog_images/ord_table.png"></center></p>
<p>Again, the most interesting thing for me is how low most of the numbers are. It&#8217;s a strange sight in a book whose thesis could be summarized as &#8220;we need to care more about existential risk&#8221;. I think most people paying attention will be <i>delighted</i> to learn there&#8217;s a 5 in 6 chance the human race will survive until 2120.</p>
<p>This is where I turn to my psychoanalysis of Toby Ord again. I think he, God help him, sees a number like that and <i>responds appropriately</i>. He multiplies 1/6th by 10 billion deaths and gets 1.6 billion deaths. Then he multiplies 1/6th by the hundreds of trillions of people it will prevent from ever existing, and gets tens of trillions of people. Then he considers that the centuries just keep adding up, until by 2X00 the risk is arbitrarily high. At that point, the difference between a 1/6 chance of humanity dying per century vs. a 5/6 chance of humanity dying may have psychological impact. But the overall takeaway from both is &#8220;Holy @!#$, we better put a lot of work into dealing with this.&#8221;</p>
<p>There&#8217;s an old joke about a science lecture. The professor says that the sun will explode in five billion years, and sees a student visibly freaking out. She asks the student what&#8217;s so scary about the sun exploding in five billion years. The student sighs with relief: &#8220;Oh, thank God! I thought you&#8217;d said five <i>million</i> years!&#8221;</p>
<p>We can imagine the opposite joke. A professor says the sun will explode in five minutes, sees a student visibly freaking out, and repeats her claim. The student, visibly relieved: &#8220;Oh, thank God! I thought you&#8217;d said five <i>seconds</i>.&#8221;</p>
<p>When read carefully, <i>The Precipice</i> is the book-length version of the second joke. Things may not be quite as disastrous as you expected. But relief may not quite be the appropriate emotion, and there&#8217;s still a lot of work to be done.</p>}
\end{xmlentriescontent}
\xmlentrieswfwcommentrss{https://slatestarcodex.com/2020/04/01/book-review-the-precipice/feed/}\xmlentriesslashcomments{508}\xmlentriestitle{SSC Journal Club: MacIntyre On Cloth Masks}
\begin{xmlentriestitledetail}
\xmltitledetailtype{text/plain}\xmltitledetailbase{https://slatestarcodex.com/feed/}\xmltitledetailvalue{SSC Journal Club: MacIntyre On Cloth Masks}
\end{xmlentriestitledetail}

\begin{xmlentrieslinks}
\xmllinksrel{alternate}\xmllinkstype{text/html}\xmllinkshref{https://slatestarcodex.com/2020/03/31/ssc-journal-club-macintyre-on-cloth-masks/}
\end{xmlentrieslinks}
\xmlentrieslink{https://slatestarcodex.com/2020/03/31/ssc-journal-club-macintyre-on-cloth-masks/}\xmlentriescomments{https://slatestarcodex.com/2020/03/31/ssc-journal-club-macintyre-on-cloth-masks/#comments}\xmlentriespublished{Wed, 01 Apr 2020 06:43:26 +0000}
\begin{xmlentriesauthors}
\xmlauthorsname{Scott Alexander}
\end{xmlentriesauthors}
\xmlentriesauthor{Scott Alexander}
\begin{xmlentriesauthordetail}
\xmlauthordetailname{Scott Alexander}
\end{xmlentriesauthordetail}

\begin{xmlentriestags}
\xmltagsterm{Uncategorized}\xmltagsterm{coronavirus}\xmltagsterm{journal club}
\end{xmlentriestags}
\xmlentriesid{https://slatestarcodex.com/?p=5918}\xmlentriessummary{[Content warning: this is a complicated analysis of something people care about a lot right now. I&#8217;m not confident in my analysis, the post comes to no clear conclusion and there are no easy answers about how to proceed. If &#8230; <a href="https://slatestarcodex.com/2020/03/31/ssc-journal-club-macintyre-on-cloth-masks/">Continue reading <span class="pjgm-metanav">&#8594;</span></a>}
\begin{xmlentriessummarydetail}
\xmlsummarydetailtype{text/html}\xmlsummarydetailbase{https://slatestarcodex.com/feed/}\xmlsummarydetailvalue{[Content warning: this is a complicated analysis of something people care about a lot right now. I&#8217;m not confident in my analysis, the post comes to no clear conclusion and there are no easy answers about how to proceed. If &#8230; <a href="https://slatestarcodex.com/2020/03/31/ssc-journal-club-macintyre-on-cloth-masks/">Continue reading <span class="pjgm-metanav">&#8594;</span></a>}
\end{xmlentriessummarydetail}

\begin{xmlentriescontent}
\xmlcontenttype{text/html}\xmlcontentbase{https://slatestarcodex.com/feed/}\xmlcontentvalue{<p><font size="1"><i>[<b>Content warning</b>: this is a complicated analysis of something people care about a lot right now. I&#8217;m not confident in my analysis, the post comes to no clear conclusion and there are no easy answers about how to proceed. If I see this on Twitter with some headline about it DESTROYING somebody, I am going to be so mad.]</i></font></p>
<p>The New York Times says that <A HREF="https://www.nytimes.com/2020/03/31/opinion/coronavirus-n95-mask.html">It&#8217;s Time To Make Your Own Face Mask</A>. But <A HREF="https://bmjopen.bmj.com/content/bmjopen/5/4/e006577.full.pdf"><b>MacIntyre et al (2015)</b></A> says it isn&#8217;t.</p>
<p>The surgical masks used in hospitals are made out of non-woven fabrics that are pretty different from anything you have at home. But in some developing countries, health care workers instead use masks made of normal cloth. Laboratory tests <A HREF="https://www.cambridge.org/core/services/aop-cambridge-core/content/view/0921A05A69A9419C862FA2F35F819D55/S1935789313000438a.pdf/testing_the_efficacy_of_homemade_masks_would_they_protect_in_an_influenza_pandemic.pdf">find</A> that improvised cloth masks block 60 &#8211; 80% of virus particles. Respirators and real surgical masks block 95%+, but 60-80% still seems better than nothing. And most of the masks ordinary people wear in Asian countries are cloth, and they seem to do pretty well. So there&#8217;s some circumstantial evidence that these cloth masks might be helpful. <A HREF="http://akkie.mods.jp/2019-nCoV/images/c/c0/%E3%82%B5%E3%83%BC%E3%82%B8%E3%82%AB%E3%83%AB%E3%83%9E%E3%82%B9%E3%82%AF%E5%8C%BB%E7%99%82vs%E5%B8%82%E4%B8%AD%E6%84%9F%E6%9F%93%E4%BA%88%E9%98%B2%EF%BC%9A%E7%B7%8F%E8%AA%AC_%282015%2C_MacIntyre%29.pdf">Most experts</A> in the early 2000s agreed that these masks were probably better than nothing. In 2015, an Australian team set out to prove it with a randomized controlled trial.</p>
<p>They went to a hospital in Vietnam and randomized workers there to a normal mask group, a cloth mask group, or a control group. Because it would have been unethical to tell the control group not to wear masks, they left the control group alone. Most control group workers did end up wearing masks sometimes, but less than the experimental groups did.</p>
<p>After a month, they counted how many infections each group had, for three different categories of infection. Here are the results:</p>
<p><center><IMG SRC="https://slatestarcodex.com/blog_images/clothmask.png"></p>
<p><font size="1"><i>Technically significant only in the ILI category, but later the authors do various post hoc adjustment for confounders and find it&#8217;s significant everywhere</i></font></center></p>
<p>For all three categories, people wearing the real surgical masks were the healthiest, the control group was in the middle, and people wearing the cloth masks were the sickest.</p>
<p>This shows real surgical masks work better than cloth masks. It&#8217;s a little bit unclear about how well cloth masks work. They do worse than the control group, but you could tell two stories about that. In one, cloth masks are worse than no mask at all. In the other, cloth masks have zero-to-slight-positive efficacy, but because some people in the control group were wearing real surgical masks some of the time, they did better than the cloth group overall. So it depends a lot what the control group was doing.</p>
<p>Unfortunately, the paper doesn&#8217;t give us all the data we want. It tells us that about 57% of both the surgical mask group and the cloth mask group wore masks regularly (defined as more than 70% of the time) but only 24% of the control group did. But there is no way of knowing whether the rest of the control group wore masks 69% of the time or 0% of the time.</p>
<p>A separate paragraph tells us that 37% of the control group used surgical masks, 8% cloth masks, and 53% used a combination of both. These numbers don&#8217;t make a lot of sense in the context of the last paragraph, so I&#8217;m going to assume they meant that <i>on the infrequent occasions they did wear masks</i>, those were the masks they used. But we don&#8217;t know if the compliant workers were disproportionately using cloth masks, disproportionately using medical masks, or both evenly. It&#8217;s hard to just eyeball these numbers and get a good sense for whether cloth masks really are worse than nothing.</p>
<p>But the authors themselves lean towards the hypothesis that that cloth masks are actively bad. First, because after some calculations I cannot quite follow, they find that the difference between surgical masks and cloth masks is so high that either the surgical masks are absurdly good, or the difference is being augmented by the cloth masks being actively bad. But nobody has previously found surgical masks to be absurdly good. The authors cite two previous studies of theirs which <i>did</i> include a no-mask control group; surgical masks did not significantly outperform nothing (they did show a trend towards doing so, and the studies were probably underpowered).</p>
<p>Second, because they compare the numbers from this study to numbers from those other two studies directly. They find the rate of infection in surgical mask users is not-significantly-different throughout the three studies, and the rate of infection in surgical mask users and no-mask controls was also not-significantly-different, and therefore surgical masks are the same as nothing and so probably the cloth masks are actively bad.</p>
<p>I am very unimpressed by this. First, you are <i>really</i> not supposed to compare things across multiple different studies. The authors protest that they did all three studies along pretty similar designs, but also admit they were different hospitals during different seasons. But second, almost no differences anywhere are significant, because all of these studies were at least a little underpowered. The current study found no significant difference between cloth masks and surgical masks in two of the three categories, even though the trend was in the expected direction. The other studies found no difference between wearing a medical mask and not wearing a medical mask, even though previous studies have suggested medical masks should work. They couldn&#8217;t even find any difference between wearing an N95 respirator and not wearing any protection at all. So when you need a chain of &#8220;x is not significantly different from y, which is not significantly different from z&#8221; in a bunch of studies that wouldn&#8217;t have been able to notice significant differences even if they existed, I stop believing it pretty quickly.</p>
<p>(In fact, I think you could use the same logic to draw the exact opposite conclusion. The cloth mask group in the current study didn&#8217;t have a significant difference from the surgical mask group in the other study, and the surgical mask group was no different from placebo, therefore cloth masks cannot have a negative effect. I find it hard to believe the authors missed this, so let me know if I am confused here.)</p>
<p>But MacIntyre et al take it seriously, and conclude:</p>
<blockquote><p>The study suggests medical masks may be protective, but the magnitude of difference raises the possibility that cloth masks cause an increase in infection risk in HCWs. Further, the filtration of the medical mask used in this trial was poor, making extremely high efficacy of medical masks unlikely, particularly given the predominant pathogen was rhinovirus, which spreads by the airborne route. Given the obligations to HCW occupational health and safety, it is important to consider the potential risk of using cloth masks [&#8230;] The physical properties of a cloth mask, reuse, the frequency and effectiveness of cleaning, and increased moisture retention, may potentially increase the infection risk for health care workers. The virus may survive on the surface of the face-masks, and modelling studies have quantified the contamination levels of masks. Self-contamination through repeated use and improper doffing is possible. For example, a contaminated cloth mask may transfer pathogen from the mask to the bare hands of the wearer. We also showed that filtration was extremely poor (almost 0%) for the cloth masks. Observations during SARS suggested double-masking and other practices increased the risk of infection because of moisture, liquid diffusion and pathogen retention. These effects may be associated with cloth masks.</p></blockquote>
<p>Why am I focusing on this one weird study so much? Because it&#8217;s the only RCT of cloth face masks we have! Millions of people, egged on by top newspapers, are about to start wearing cloth face masks during a pandemic, when right now the authors of the only randomized trial on them conclude they&#8217;re probably net harmful. This should be really scary! Somebody with more experience and statistical knowledge than I have should be looking this over with a fine-toothed comb and trying to figure out what we should do.</p>
<p>Until then, should people stay away from cloth masks? I&#8217;m not sure, and this is <i>so</i> not a recommendation, but I lean toward no. The prior that they should work or at least be neutral is too high for a study this weak to convince me otherwise. <b>More important, this study only examines incoming pathogens. Even if they are harmful for blocking incoming pathogens, there are still reasons to think they are helpful for blocking outgoing ones.</b> If I had to hang out with a coronavirus patient for a while, and I had to choose between both of us wearing cloth masks, or neither, I would go with the masks. Only until we could get real surgical masks, which are much better. But I&#8217;d go with the cloth ones instead of nothing.</p>
<p>But right now that&#8217;s a gut judgment, and the evidence says I&#8217;m wrong. This is one of those times where people have to make a life-or-death decision in conditions of high uncertainty, and it really sucks.</p>
<p>[EDIT: Bolded a passage I think is important to make sure people don&#8217;t miss it]</p>}
\end{xmlentriescontent}
\xmlentrieswfwcommentrss{https://slatestarcodex.com/2020/03/31/ssc-journal-club-macintyre-on-cloth-masks/feed/}\xmlentriesslashcomments{126}\xmlentriestitle{Legal Systems Very Different From Ours, Because I Just Made Them Up}
\begin{xmlentriestitledetail}
\xmltitledetailtype{text/plain}\xmltitledetailbase{https://slatestarcodex.com/feed/}\xmltitledetailvalue{Legal Systems Very Different From Ours, Because I Just Made Them Up}
\end{xmlentriestitledetail}

\begin{xmlentrieslinks}
\xmllinksrel{alternate}\xmllinkstype{text/html}\xmllinkshref{https://slatestarcodex.com/2020/03/30/legal-systems-very-different-from-ours-because-i-just-made-them-up/}
\end{xmlentrieslinks}
\xmlentrieslink{https://slatestarcodex.com/2020/03/30/legal-systems-very-different-from-ours-because-i-just-made-them-up/}\xmlentriescomments{https://slatestarcodex.com/2020/03/30/legal-systems-very-different-from-ours-because-i-just-made-them-up/#comments}\xmlentriespublished{Tue, 31 Mar 2020 00:54:38 +0000}
\begin{xmlentriesauthors}
\xmlauthorsname{Scott Alexander}
\end{xmlentriesauthors}
\xmlentriesauthor{Scott Alexander}
\begin{xmlentriesauthordetail}
\xmlauthordetailname{Scott Alexander}
\end{xmlentriesauthordetail}

\begin{xmlentriestags}
\xmltagsterm{Uncategorized}\xmltagsterm{fiction}
\end{xmlentriestags}
\xmlentriesid{https://slatestarcodex.com/?p=5917}\xmlentriessummary{[with apologies to the real Legal Systems Very Different From Ours. See also the List Of Fictional Drugs Banned By The FDA] I. The Clamzorians are animists. They believe every rock and tree and river has its own spirit. And &#8230; <a href="https://slatestarcodex.com/2020/03/30/legal-systems-very-different-from-ours-because-i-just-made-them-up/">Continue reading <span class="pjgm-metanav">&#8594;</span></a>}
\begin{xmlentriessummarydetail}
\xmlsummarydetailtype{text/html}\xmlsummarydetailbase{https://slatestarcodex.com/feed/}\xmlsummarydetailvalue{[with apologies to the real Legal Systems Very Different From Ours. See also the List Of Fictional Drugs Banned By The FDA] I. The Clamzorians are animists. They believe every rock and tree and river has its own spirit. And &#8230; <a href="https://slatestarcodex.com/2020/03/30/legal-systems-very-different-from-ours-because-i-just-made-them-up/">Continue reading <span class="pjgm-metanav">&#8594;</span></a>}
\end{xmlentriessummarydetail}

\begin{xmlentriescontent}
\xmlcontenttype{text/html}\xmlcontentbase{https://slatestarcodex.com/feed/}\xmlcontentvalue{<p><font size="1"><i>[with apologies to the real <A HREF="https://slatestarcodex.com/2017/11/13/book-review-legal-systems-very-different-from-ours/">Legal Systems Very Different From Ours</A>. See also the <A HREF="https://slatestarcodex.com/2013/10/25/list-of-fictional-drugs-banned-by-the-fda/">List Of Fictional Drugs Banned By The FDA</A>]</i></font></p>
<p><b>I.</b></p>
<p>The Clamzorians are animists. They believe every rock and tree and river has its own spirit. And those spirits are legal people. This on its own is not unusual &#8211; <A HREF="http://www.bbc.com/travel/story/20200319-the-new-zealand-river-that-became-a-legal-person">even New Zealand</A> gives rivers legal personhood. But in Clamzoria, if a flood destroys your home, you sue the river.</p>
<p>If you win, then the river is in debt to you. The government can assign a guardian to the river to force it to pay off its debts, and that guardian gets temporary custody of all the river&#8217;s property. He or she can collect a toll from boats, sell water to reservoirs, and charge rent to hydroelectric dams. Once the river has paid off its debt, the guardian is discharged, and the river becomes free to use once again.</p>
<p>Clamzorian precedent governs when you may or may not sue objects. If you swim in the freezing river in the dead of winter, and catch cold, that&#8217;s on you. But if a hurricane destroys your property, you can absolutely sue the wind for damages, and collect from windmills. Suits against earthquakes, volcanoes, and the like are dead common. Suits against diseases happen occasionally. Sometimes someone will sue something even more abstract &#8211; a custom, an emotion, a concept.</p>
<p>Legend tells of a lawyer who once sued Death itself for wrongful death, a class action suit on behalf of everyone who ever lived. The judge found in favor of the plaintiff, but the appointed custodian despaired at ever collecting the judgment &#8211; the few morticians and undertakers in the realm couldn&#8217;t afford even a fraction of the damages. In a stroke of genius, he went after the military, and charged them for the right to kill enemy soldiers. The military grumbled, but eventually gave in: fair is fair.</p>
<p><b>II.</b></p>
<p>Fixed fines are inherently unfair to the poor. If you fine people $50 for running a red light, you&#8217;ve charged someone who makes $10,000 0.5% of their income, but someone who makes $100,000 gets off with only 0.05% of their income.</p>
<p>But prison sentences are inherently unfair to the rich. After all, if you already live in a crowded slum much like a prison cell, and your life is prison-level boring and oppressive already, then going to prison barely costs you anything. But if you live in a mansion and spend all day indulging in the finest luxuries on offer, going to prison is a massive decrease in your quality of life.</p>
<p>The people of Pohjankaupunki thought long and hard about this problem, and came up with a solution: crimes will be punished by neither fines nor prison. They will be punished by government mandated prescription of rimonabant, a prodepressant medication which directly saps your ability to feel happiness. Running a red light may get you 5 mg rimonabant for a month. Murder may get you 80 mg rimonabant twice a day for ten years.</p>
<p>There is no capital punishment in Pohjankaupunki, but if a criminal decides to commit suicide rather than continue to take their medication, they are considered to have voluntarily upgraded to the death penalty, and their debt to the state has been repaid.</p>
<p><b>III.</b></p>
<p>Sloviria is an enlightened country. They do not blame criminals for their actions. They realize it is Society&#8217;s fault for making criminals that way. So when someone commits a crime, they punish Society.</p>
<p>Sloviria is very technologically advanced, with plenty of social networking sites and GPS tracking of cell phones and all the other systems that create a nice objective social graph. When someone commits a crime, the government lets them go free, and punishes everyone else, in proportion to how close they were to the offender on the social graph. If the punishment for a certain crime is a $1000 fine, perhaps each of their parents and their partner pays $200, their boss and best friend pay $100, some of their teachers a few tenners each, and more distant friends and relations a few dollars or less. If a friend of a friend who you met at a dinner party once commits murder, you may be out a couple of cents.</p>
<p>This isn&#8217;t to say perpetrators get off scot-free; Sloviria isn&#8217;t <i>that</i> enlightened. The punishment for perpetrators is that nobody wants to interact with them, for fear that they might perpetrate again. Once a person is a known criminal &#8211; or a suspected criminal, or just the sort of person who seems like they might become a criminal &#8211; their friends, families, and business relations shun them, trying to minimize their potential loss. This threat alone is enough to discourage crime and every form of crime-adjacent misbehavior.</p>
<p>The Slovirian Radical Party is even more enlightened than Sloviria as a whole, and opposes social punishment. They believe that such punishment prevents rehabilitation, since criminals and at-risk youth find it impossible to make the connections they need to succeed, and are forced to hang out with other people as criminal as themselves. They propose a complete inversion of Sloviria&#8217;s justice system; when anyone commits a crime, the people closest to them are <i>rewarded</i>. They envision a future where, once somebody shows any sign of being at risk for antisocial behavior, they are love-bombed by dozens of people hoping to get rich off their acquaintance, people who want to employ them, adopt them, date them, or just serve as mentors and parental figures. But wouldn&#8217;t all these people encourage the potential criminal to offend? The Radicals debate this among themselves, with one solution being that <i>this</i> could just be a perfectly normal crime punished by jail time.</p>
<p><b>IV.</b></p>
<p>Nova-Nishistan&#8217;s legal system is based on blackmail. It&#8217;s not <i>just</i> blackmail. There are courts and jails and so on. But few people use them. If you have evidence that someone committed a crime, you are expected to threaten to report them unless they give you money.</p>
<p>The system has many advantages. The person most likely to have evidence of a crime is the victim. The victim can choose how much money they want as damages, and have a good chance of receiving it. Fines are automatically calibrated to the wealth of the victim, so poor people are not stuck with debts that are impossible to pay. If a crime is victimless, or the victim chooses not to prosecute, any other witnesses are incentivized to take up the cause of punishing the wrongdoer of their own initiative. Few crimes make it to the courts or prisons, so everyone is assured a speedy trial and an jail cell free of overcrowding. </p>
<p>In order to maintain their system, the Nova-Nishistanis need many laws related to blackmail itself. One of their most serious crimes is to blackmail someone, receive the requested ransom, but report them anyway; anyone convicted of this will be in for a lengthy prison sentence. Indefinite blackmail &#8211; &#8220;pay me $100 now, but I might ask for more later&#8221; &#8211; is forbidden. So is non-monetary blackmail; too easy to abuse. There are a host of similar regulations.</p>
<p>One regulation they don&#8217;t need is laws about retaliating against blackmailers. You might expect this to be a problem &#8211; blackmailing the mob sounds pretty scary. But there are lots of individuals, companies, and (let&#8217;s face it) rival gangs happy to provide dead-man&#8217;s-switch-as-a-service. Tell them your secret (which they promise not to disclose without your consent), and if anything happens to you, they prosecute it. Even better, if anything happens to you, they&#8217;re almost guaranteed to investigate your death, since their special evidence gives them a leg up in what could be a very lucrative blackmail case.</p>
<p>Of course, this only works on people who are rational enough to respond to incentives. If someone is a complete unpredictable psycho, you probably don&#8217;t want to try blackmailing them, even with a dead-man&#8217;s-switch as insurance. But these are probably the people who should be in jail anyway!</p>
<p><b>V.</b></p>
<p>The people of Bogolia thought it was unfair that rich people could hire better lawyers than poor people. But they didn&#8217;t want to take the authoritarian step of banning rich people from buying good lawyers, if they thought skilled representation was important. Instead, they just mandated that in any legal case, both sides had to have equally-priced counsel. A rich person could hire as expensive a defense attorney as they wanted, as long as they donated an equal sum to the plaintiff to hire star attorneys of their own. You could sue someone with as highly-priced an attorney as you wanted, but you needed to give them the same amount to spend on their defense.</p>
<p>(this rule applied to the state too, and so implied the right to a public defender worth however much the state was paying to prosecute you, even if you were poor and couldn&#8217;t otherwise afford one)</p>
<p>Some trolls tried launching hundreds of frivolous lawsuits against companies they didn&#8217;t like, assuming that the company would have to pay both sides of the lawsuit and eventually go broke. They were punished through the normal anti-frivolous-lawsuit rules, and it turned out that companies that did not go broke having to pay one side of a lawsuit don&#8217;t go broke having to pay both sides either.</p>
<p>But there were some weirder unintended consequences. How good a lawyer to get became a highly strategic decision for rich clients facing poorer ones. If you thought you were in the right, you&#8217;d get a good lawyer, since two equally good lawyers facing off will likely produce truth. If you thought you were in the wrong, you&#8217;d try to get a crappy lawyer, since then your opponent would also have a crappy lawyer, and two crappy lawyers facing off will likely produce random results. Not paying for a good lawyer started to be seen as an admission that one&#8217;s case was weak.</p>
<p>But also, lawyer salaries started to get wacky. If a random criminal hurt a rich person somehow, and the rich person hired a good lawyer, the random criminal might receive tens of thousands of dollars to spend on legal advice. But random criminals generally are not savvy at evaluating lawyer skill, so thousands of predatory lawyers sprang up, willing to cater to these people by looking impressive and accepting very high salaries. For the savviest of political operators, an equal and opposite caste of underpriced lawyers sprang up, who would accept very low pay in exchange for vague social credit to be doled out later. More and more political scandals started to center on prestigious lawyers defending politicians for free in exchange for favors, and so depriving the opposing party of their right to equally-matched counsel.</p>
<p>Finally the authorities handed down a change to the system: the plaintiff and defendant would agree on two lawyers to conduct the trial. Then the judge would flip a coin, and one of the two would be assigned at random to each party.</p>
<p><b>VI.</b></p>
<p>Sanzorre accidentally became <A HREF="https://slatestarcodex.com/2015/03/18/book-review-the-machinery-of-freedom/">an anarcho-capitalist state</A> under the dominion of malpractice insurance companies.</p>
<p>They started off by insuring doctors. Doctors know a bad malpractice case could ruin them. And although being a good doctor helps, it&#8217;s not 100%. Even the best doctor can get unlucky, or have somebody with a grudge fabricate a case against them. For that matter, even very bad doctors can get lucky and never have to deal with a case at all. So doctors have malpractice insurance, and if they seem to be practicing medicine badly their insurance company will raise their premiums.</p>
<p>This worked well enough that other industries started adopting it too. If a factory&#8217;s pollutant byproducts got discovered to cause cancer ten years later, their industrial malpractice insurance would pay for it. If someone slipped and fell and broke their back on a restaurant floor, their restaurant malpractice insurance would pay for it. Of course, these insurance companies worked closely with factories to monitor how many they were polluting, and gave discounts to restaurants which followed best practices on floor cleaning.</p>
<p>Finally, they branched out to serving ordinary people. If you accidentally hit someone&#8217;s dog with your car and got sued for damages, better to have a personal malpractice insurance pay them than get hit for tens of thousands of dollars yourself. Having malpractice insurance became to Sanzorrians what having health insurance is to Americans &#8211; a necessity if you don&#8217;t want to court disaster.</p>
<p>The plaintiffs in all these cases were usually being covered by lawyers who took contigency fees. But as malpractice insurance companies became better at their jobs, the contingency fees began to dry up. Finally, lobbyists from the insurance companies got contingency fees banned entirely. This presented a dilemma for ordinary people with grievances against bad actors. Thus the rise of the grievance insurance.</p>
<p>If you suffered harm from a doctor&#8217;s medical error, and had grievance insurance, the insurance company would pay the cost of the malpractice suit. If you were poisoned by industrial runoff, the insurance company would pay the cost of suing the factory. Grievance insurance soon became as essential as malpractice insurance. Without it, you wouldn&#8217;t be able to stand up for your rights.</p>
<p>Like malpractice insurance, grievance insurance was only available cheaply to people who agreed to avoid risks. If you wanted to be able to sue for malpractice, you had to avoid going to quacks. If you wanted to be able to sue factories for pollution, you couldn&#8217;t live right next to a coal plant. Gradually, grievance insurances placed more and more restrictions on people&#8217;s behavior, and people generally complied.</p>
<p>As malpractice insurances incentivized potential defendants to avoid actions that could harm others, and grievance insurances incentivized individuals to avoid risk, the number of lawsuits gradually got fewer and fewer. Those that happened were generally settled between malpractice insurers and grievance insurers, without ever having to go to court, and sometimes with both companies changing their policy to avoid repeats in the future. Soon, even this formality was eliminated &#8211; each malpractice insurance company paid a negotiated amount to each grievance insurance company each year, and the grievance insurance company paid complainants from its own bank account as per its own policies whenever they complained.</p>
<p>It wasn&#8217;t quite full anarcho-capitalism. The state still intervened in a few very serious crimes, like murder. But the insurance companies had replaced the civil courts and the regulatory apparatus, and controlled every aspect of doing business.</p>
<p><b>VII.</b></p>
<p>Modern philosophy says that formal systems are bunk. The dream of reducing the complexity of reality to some mere set of rules is a childish desire reminiscent of the fascists and high modernists of the early 20th century. Enlightened thinkers realize that we need a Kegan 5 type fluid ability to transcend systematicity. So the people of Mirakoth don&#8217;t have laws. They&#8217;re just supposed to not do bad stuff.</p>
<p>If someone in Mirakoth thinks someone else did something bad, they can bring it before a council of seven judges. If a majority of the judges think it was bad, they can assign whatever seems to them like fair punishment. If the loser appeals, it goes to a larger council of forty-nine judges. If they think it was bad, it was bad. These judges are under no obligation to follow precedent or any particular philosophy. They&#8217;re just supposed to be in favor of good stuff and against bad stuff.</p>
<p>In order to prevent people from seeking out judges who agree with them, each case is assigned seven judges at random. All cases are tried by videoconference, to make sure the judge pool is unlimited by geographical mobility. If the judges think a case is frivolous, they can choose to punish the person who brought the case.</p>
<p>Doesn&#8217;t this create such paralyzing uncertainty that nobody knows if they can do anything at all? Not really. Controversial cases are more likely to go to the full 49 judge panel. If an opinion is only held by 20% of judges in the country, then there&#8217;s only about a 1 in a million chance that the panel will rule in favor. Even if the opinion is held by 40%, it&#8217;s still only an 8% chance of winning. So just don&#8217;t do things that more than 40% of people think are bad, and you&#8217;ll be fine!</p>}
\end{xmlentriescontent}
\xmlentrieswfwcommentrss{https://slatestarcodex.com/2020/03/30/legal-systems-very-different-from-ours-because-i-just-made-them-up/feed/}\xmlentriesslashcomments{93}\xmlentriestitle{Open Thread 150.5}
\begin{xmlentriestitledetail}
\xmltitledetailtype{text/plain}\xmltitledetailbase{https://slatestarcodex.com/feed/}\xmltitledetailvalue{Open Thread 150.5}
\end{xmlentriestitledetail}

\begin{xmlentrieslinks}
\xmllinksrel{alternate}\xmllinkstype{text/html}\xmllinkshref{https://slatestarcodex.com/2020/03/29/open-thread-150-5/}
\end{xmlentrieslinks}
\xmlentrieslink{https://slatestarcodex.com/2020/03/29/open-thread-150-5/}\xmlentriescomments{https://slatestarcodex.com/2020/03/29/open-thread-150-5/#comments}\xmlentriespublished{Sun, 29 Mar 2020 16:00:43 +0000}
\begin{xmlentriesauthors}
\xmlauthorsname{a reader}
\end{xmlentriesauthors}
\xmlentriesauthor{a reader}
\begin{xmlentriesauthordetail}
\xmlauthordetailname{a reader}
\end{xmlentriesauthordetail}

\begin{xmlentriestags}
\xmltagsterm{Uncategorized}\xmltagsterm{open}
\end{xmlentriestags}
\xmlentriesid{https://slatestarcodex.com/?p=5743}\xmlentriessummary{This is would usually be a hidden open thread, but I&#8217;m promoting it to front page to say a couple of things: 1. Future of Humanity Institute asks me to advertise their free pandemic modeling software for hospitals, policymakers, and &#8230; <a href="https://slatestarcodex.com/2020/03/29/open-thread-150-5/">Continue reading <span class="pjgm-metanav">&#8594;</span></a>}
\begin{xmlentriessummarydetail}
\xmlsummarydetailtype{text/html}\xmlsummarydetailbase{https://slatestarcodex.com/feed/}\xmlsummarydetailvalue{This is would usually be a hidden open thread, but I&#8217;m promoting it to front page to say a couple of things: 1. Future of Humanity Institute asks me to advertise their free pandemic modeling software for hospitals, policymakers, and &#8230; <a href="https://slatestarcodex.com/2020/03/29/open-thread-150-5/">Continue reading <span class="pjgm-metanav">&#8594;</span></a>}
\end{xmlentriessummarydetail}

\begin{xmlentriescontent}
\xmlcontenttype{text/html}\xmlcontentbase{https://slatestarcodex.com/feed/}\xmlcontentvalue{<p>This is would usually be a hidden open thread, but I&#8217;m promoting it to front page to say a couple of things:</p>
<p><b>1.</b> Future of Humanity Institute asks me to advertise their <A HREF="http://epidemicforecasting.org/">free pandemic modeling software</A> for hospitals, policymakers, and anyone who just wants to play around with free pandemic modeling software.</p>
<p><b>2.</b> People seem to be confused whether my <A HREF="https://slatestarcodex.com/2020/03/23/face-masks-much-more-than-you-wanted-to-know/">face masks</A> post last week was coming out against or in favor of face masks. Although it&#8217;s a complicated issue, I meant for it to conclude that (modulo the importance of reserving them for health care personnel), wearing face masks is probably helpful. </p>
<p><b>3.</b> On Friday, I stated that people should stop smoking to reduce their risk of serious lung complications of coronavirus. Although that conclusion was supported by one Chinese study and by common sense, a few people have pointed out to me that more recent studies show the opposite. <A HREF="https://www.qeios.com/read/article/550">This study</A> of Chinese patients finds that smoking and vaping are not dangerous in coronavirus and may have &#8220;a protective role&#8221;, possibly due to downregulation of ACE (but note that the lead author has <A HREF="https://tobacco.ucsf.edu/surprise-lorillard-tobacco-publishes-two-papers-finding-e-cigs-pose-no-hazard">a history</A> of getting funding from e-cigarette companies). <A HREF="https://www.nejm.org/doi/full/10.1056/NEJMoa2002032">This study</A> from China finds that although never-smokers have better survival rates than current-smokers, former-smokers do worse than either, which would argue against quitting right now. And <A HREF="https://www.preprints.org/manuscript/202002.0408/v1">this study</A> confirms that quitting smoking can upregulate expression of coronavirus receptor genes (though it finds that smoking does as well).</p>
<p>I&#8217;m pretty suspicious of this research. It&#8217;s new, lots of it isn&#8217;t yet peer reviewed, and it contradicts itself in places. The former-smokers-do-worse effect is reminiscent of the teetotalers-do-worst effect in alcohol research, which is probably because very sick people get told to stop drinking, and so teetotalers are a disproportionately sickly population. Everything is working off a few heavily-biased mortality numbers in China. And also, even if quitting smoking increases your coronavirus mortality risk it will still be very good for you on net. </p>
<p>Still, the most recent research <i>does</i> apparently show that the advice I gave you yesterday was diametrically wrong and could kill you, so I figured I had better get that out there.</p>}
\end{xmlentriescontent}
\xmlentrieswfwcommentrss{https://slatestarcodex.com/2020/03/29/open-thread-150-5/feed/}\xmlentriesslashcomments{1430}\xmlentriestitle{Coronalinks 3/27/20: We’re Number One}
\begin{xmlentriestitledetail}
\xmltitledetailtype{text/plain}\xmltitledetailbase{https://slatestarcodex.com/feed/}\xmltitledetailvalue{Coronalinks 3/27/20: We’re Number One}
\end{xmlentriestitledetail}

\begin{xmlentrieslinks}
\xmllinksrel{alternate}\xmllinkstype{text/html}\xmllinkshref{https://slatestarcodex.com/2020/03/27/coronalinks-3-27-20/}
\end{xmlentrieslinks}
\xmlentrieslink{https://slatestarcodex.com/2020/03/27/coronalinks-3-27-20/}\xmlentriescomments{https://slatestarcodex.com/2020/03/27/coronalinks-3-27-20/#comments}\xmlentriespublished{Fri, 27 Mar 2020 09:46:14 +0000}
\begin{xmlentriesauthors}
\xmlauthorsname{Scott Alexander}
\end{xmlentriesauthors}
\xmlentriesauthor{Scott Alexander}
\begin{xmlentriesauthordetail}
\xmlauthordetailname{Scott Alexander}
\end{xmlentriesauthordetail}

\begin{xmlentriestags}
\xmltagsterm{Uncategorized}\xmltagsterm{coronavirus}\xmltagsterm{links}
\end{xmlentriestags}
\xmlentriesid{https://slatestarcodex.com/?p=5914}\xmlentriessummary{The United States now has more coronavirus cases than any other country, including China, marking a new stage in the epidemic. As before, feel free to treat this as an open thread for all coronavirus-related issues. Everything here is speculative &#8230; <a href="https://slatestarcodex.com/2020/03/27/coronalinks-3-27-20/">Continue reading <span class="pjgm-metanav">&#8594;</span></a>}
\begin{xmlentriessummarydetail}
\xmlsummarydetailtype{text/html}\xmlsummarydetailbase{https://slatestarcodex.com/feed/}\xmlsummarydetailvalue{The United States now has more coronavirus cases than any other country, including China, marking a new stage in the epidemic. As before, feel free to treat this as an open thread for all coronavirus-related issues. Everything here is speculative &#8230; <a href="https://slatestarcodex.com/2020/03/27/coronalinks-3-27-20/">Continue reading <span class="pjgm-metanav">&#8594;</span></a>}
\end{xmlentriessummarydetail}

\begin{xmlentriescontent}
\xmlcontenttype{text/html}\xmlcontentbase{https://slatestarcodex.com/feed/}\xmlcontentvalue{<p>The United States now has <A HREF="https://www.vox.com/2020/3/26/21194153/us-confirmed-coronavirus-cases-world">more coronavirus cases than any other country, including China</A>, marking a new stage in the epidemic. As before, feel free to treat this as an open thread for all coronavirus-related issues. Everything here is speculative and not intended as medical advice.</p>
<p><b>Hammer and dance</b></p>
<p>Most of the smart people I&#8217;ve been reading have converged on something like the ideas expressed in <A HREF="https://medium.com/@tomaspueyo/coronavirus-the-hammer-and-the-dance-be9337092b56">The Hammer And The Dance</A> &#8211; see <A HREF="https://www.lesswrong.com/posts/Ddgry4k64oBZYfrHy/covid-19-points-of-leverage-travel-bans-and-eradication">this Less Wrong post</A> for more. </p>
<p>Summary: Asian countries have managed to control the pandemic through mass testing, contact tracing, and travel bans, without economic shutdown. The West lost the chance for a clean win when it bungled its first month of response, but it can still recover its footing. We need a medium-term national shutdown to arrest the spread of the virus until authorities can get their act together &#8211; manufacture lots of tests and face masks, create a testing infrastructure, come up with policies for how to respond when people test positive, distribute the face masks to everyone, etc. With a lot of work, we can manage that in a month or so. After that, we can relax the national shutdown, start over with a clean slate, and pursue the Asian-style containment strategy we should have been doing since the beginning.</p>
<p>This is the only plan I&#8217;ve heard from anybody that doesn&#8217;t result in either hundreds of thousands of deaths, or the economy crashing so hard we&#8217;re all reduced to eating weeds and rocks. </p>
<p>I relayed some criticism of a previous Medium post, <A HREF="https://medium.com/@joschabach/flattening-the-curve-is-a-deadly-delusion-eea324fe9727">Flattening The Curve Is A Deadly Delusion</A>, last links post. In retrospect, I was wrong, it was right (except for the minor math errors it admitted to), and it was trying to say something similar to this. There is no practical way to &#8220;flatten the curve&#8221; except by making it so flat that the virus is all-but-gone, like it is in South Korea right now. I think this was also the conclusion of the <A HREF="https://www.nytimes.com/2020/03/17/world/europe/coronavirus-imperial-college-johnson.html">Imperial College London report</A> that everyone has been talking about.</p>
<p><b>Thank you for not smoking</b></p>
<p>There isn&#8217;t a lot you can do to improve your chances if you get coronavirus, but one really important intervention you can take right now is to STOP SMOKING.</p>
<p>I try not to lecture my patients on their health failings. I am not a jerk to obese people or people who don&#8217;t get enough exercise. But I try to tell every smoker, at least once, to STOP SMOKING. Studies have shown that having a doctor or other authority figure say this <A HREF="https://academic.oup.com/ntr/article/20/12/1418/4160115">actually helps a lot</A>, and every person who STOPS SMOKING <A HREF="https://well.blogs.nytimes.com/2013/01/23/putting-a-number-to-smokings-toll/">gains 5 &#8211; 10 years</A> of life expectancy. There is nothing else you can do as a doctor or a human being that gives you a medium chance of saving ten life-years with a ten second speech. Everything that effective altruism has to offer pales in comparison. So even though I hate lecturing people &#8211; on this blog as much as in my medical practice &#8211; I suck it up and tell everyone STOP SMOKING.</p>
<p>If you need a reason to quit now instead of later, here it is: coronavirus is a lot worse for smokers. The virus kills by infecting your lungs. If your respiratory health is pretty good, you have lung capacity to spare and will probably be okay. If your respiratory health is already iffy, you will need ventilation and maybe die. From <A HREF="https://coronawiki.org/page/covid-19-the-role-of-smoking-cessation-during-respiratory-virus-epidemics">this article</A>:</p>
<blockquote><p>An article reporting disease outcomes in 1,099 laboratory confirmed cases of covid-19 reported that 12.4% (17/137) of current smokers died, required intensive care unit admission or mechanical ventilation compared with 4.7% (44/927) among never smokers. Smoking prevalence among men in China is approximately 48% but only 3% in women; this is coupled with findings from the WHO-China Joint Mission on Coronavirus Disease 2019, which reports a higher case fatality rate among males compared with females (4.7% vs. 2.8%).</p></blockquote>
<p>[EDIT: In Sweden, men and women smoke equally <A HREF="https://www.svt.se/nyheter/inrikes/nastan-8-av-10-av-de-svarast-sjuka-i-sverige-ar-man">but men still die more</A>, so the gender argument may not be as strong as it sounded a few weeks ago]</p>
<p>I want to clarify that what I&#8217;m telling you right now is totally unprincipled propaganda, intended to take advantage of a moment of panic &#8211; realistically, on the list of ways smoking can kill you, coronavirus is somewhere near the bottom. Quick back-of-the-napkin math: assume you have a 30% chance of getting coronavirus this year, that smokers&#8217; death rate is 4% compared to non-smokers&#8217; 1%, so quitting smoking now will save you a 1% risk of coronavirus death this year. But about 10% of smokers get lung cancer eventually, compared to very few non-smokers, and lung cancer has about a 66% death rate, so it&#8217;ll save you a 6.6% chance of death by lung cancer. Honestly, coronavirus shouldn&#8217;t even figure into your calculations here. </p>
<p>But since you <i>are</i> panicking about coronavirus right now, you might as well use it as motivation to STOP SMOKING. Smokers&#8217; lungs start to heal <A HREF="https://www.medicalnewstoday.com/articles/317956">as soon as one month</A> after quitting &#8211; so quit now, and if Trump makes good on his threats to stop self-isolation and restart the epidemic after Easter, you&#8217;ll be feeling better by the time things get bad again.</p>
<p>Some people have a lot of trouble quitting smoking. If you&#8217;ve been unsuccessful before and you don&#8217;t have good access to medical care, try e-cigarettes &#8211; whatever you&#8217;ve heard about them, they&#8217;re infinitely better than the real thing. If you do have good access to medical care, ask your doctor for bupropion (aka &#8220;Wellbutrin&#8221;, &#8220;Zyban&#8221;), a very effective stop-smoking medication. I have seen dozens of patients quit smoking on bupropion; my most recent success was yesterday. It&#8217;s a great medication, and the most common side effects are <A HREF="https://docs.google.com/document/d/1niiV8I4cgk_xZ1Blou15ImPmqXU4eb_li9eRVp5NgYo/edit">curing your depression</A>, <A HREF="https://www.harpersbazaar.com/beauty/health/a1631/the-happy-sexy-skinny-pill/">improving your sex life</A>, and <A HREF="https://www.medscape.com/viewarticle/863364">making you lose weight</A>. If you&#8217;re worried about going outside to get it, remember that most US doctors and psychiatrists are seeing people by video now, and many pharmacies have started drive-thru and delivery services. Alternately, you could travel to your local pharmacy on a crowded bus, lick everyone who goes on or off, then stop in Wuhan on your way home for a tasty bowl of bat soup. It doesn&#8217;t matter, taking care of this now instead of putting it off <i>would still increase your life expectancy on net</i>.</p>
<p><b>Japan and other mysteries</b></p>
<p>Japan should be having a terrible time right now. They were one of the first countries to get coronavirus cases, around the same time as South Korea and Italy. And their response has been somewhere between terrible and nonexistent. A friend living in Japan <A HREF="https://serinemolecule.tumblr.com/post/613236399346434048/how-are-things-in-japan-under-coronavirus-did">says that</A> &#8220;Japan has the worst coronavirus response in the world (the USA is second worst)&#8221;, and gets backup from commenters, including a photo of still-packed rush hour trains. Japan is super-dense and full of old people, so at this point the living should envy the dead.</p>
<p>But actually their case number has barely budged over the past month. It was 200 a month ago. Now it&#8217;s 1300. This is the most successful coronavirus containment by any major country&#8217;s, much better than even South Korea&#8217;s, and it was all done with zero effort.</p>
<p>The obvious conclusion is that Japan just isn&#8217;t testing anyone. This turns out to be true &#8211; they were hoping that if they made themselves look virus-free, the world would still let them hold the Tokyo Olympics this summer.</p>
<p>But at this point, it should be beyond their ability to cover up. We should be getting the same horrifying stories of overflowing hospitals and convoys of coffins that we hear out of Italy. Japanese cities should be defying the national government&#8217;s orders and going into total lockdowns. Since none of this is happening, it looks like Japan really is almost virus-free. <A HREF="https://www.japantimes.co.jp/opinion/2020/03/21/commentary/japan-commentary/japan-still-coronavirus-outlier/">The Japan Times</A> is as confused about this as I am.</p>
<p>Some people have gestured at the Japanese being an unusually clean and law-abiding people. Maybe the government has just sort of subtly communicated &#8220;don&#8217;t do anything that will mess up our Olympics chances&#8221; and everyone has been really good at not touching their face. Maybe widespread use of face masks is much <i>much</i> more important than anyone has previously believed. I don&#8217;t know.</p>
<p>One way this should affect us Westerners is by making us worried that an Asian-style containment strategy wouldn&#8217;t work here. The evidence in favor of such a strategy is that it worked in a bunch of Asian countries like South Korea, Taiwan, Hong Kong, and Singapore. But if there&#8217;s something about wealthy orderly mask-wearing Asian societies that makes them mysteriously immune to the pandemic, maybe their containment strategies aren&#8217;t really that impressive. Maybe they just needed a little bit of containment to tip them over the edge. I don&#8217;t know, things sure seemed bad in South Korea a few weeks ago (and in Wuhan). I am so boggled by this that I don&#8217;t know what to think.</p>
<p>Also, what about Iran? The reports sounded basically apocalyptic a few weeks ago. They stubbornly refused to institute any lockdowns or stop kissing their sacred shrines. Now they have fewer cases than Spain, Germany, or the US. A quick look at the data confirms that their doubling time is now 11 days, compared to six days in Italy and four in the US. Again, I have no explanation.</p>
<p><b>Takeout</b></p>
<p>So far every US state and local self-isolation order has included exceptions for getting takeout or delivery food. I&#8217;m sure restaurants appreciate the business and consumers appreciate getting to keep that particular aspect of a normal lifestyle. But is it actually safe?</p>
<p>All the big organizations say yes. From <A HREF="https://www.forbes.com/sites/victoriaforster/2020/03/25/is-eating-takeout-food-safe-during-the-coronavirus-pandemic/">Forbes</A>:</p>
<blockquote><p>“Takeout food seems to pose a very minimal risk of passing on coronavirus. Here, virology experts explain why&#8230;.&#8221;There is no evidence that SARS-CoV-2 can be transmitted by eating food. I imagine that if this is possible, the risk is extremely low,” said Angela L. Rasmussen, PhD, a virologist in the faculty of the Center for Infection and Immunity at the Columbia Mailman School of Public Health, adding that she is not aware of any human coronaviruses that can be transmitted through food.</p></blockquote>
<p>And the <A HREF="https://www.sfchronicle.com/coronavirus/article/is-take-out-and-delivery-food-safe-coronavirus-15152786.php">San Francisco Chronicle</A>:</p>
<blockquote><p>With dining in restaurants off the table, many Americans are wondering if take-out and delivery food options are still viable in the age of coronavirus. Luckily for people tired of their own home cooking, the answer is, by and large, yes.</p>
<p>According to the CDC, transmission of COVID-19 primarily happens person-to-person, so your largest risk is not in the food but in human interaction. Keep your distance as much as possible when picking up food, or request that delivery workers leave the food on your doorstep. As with other in-person interactions, remember to avoid touching your face and be sure to wash your hands thoroughly as soon as you can.</p>
<p>&#8220;It may be possible that a person can get COVID-19 by touching a surface or object that has the virus on it and then touching their own mouth, nose, or possibly their eyes, but this is not thought to be the main way the virus spreads,&#8221; the CDC says.</p></blockquote>
<p>On the other hand, all of my friends who are actually worried about getting the condition are avoiding delivery food like, well, the plague. Their argument is that we know the virus can survive on surfaces for a while, so all you need is one food worker to cough on your food after it&#8217;s been cooked (or on food that doesn&#8217;t get cooked at all), and you&#8217;re screwed. Restauarants are supposed to follow sanitary precautions, but people familiar with the industry say these precautions are not so strong to 100% (or even an especially high percent) ensure you get un-coughed-on food. The CDC <A HREF="https://www.mercurynews.com/2020/03/16/coronavirus-cdc-answers-questions-on-safety-of-take-out-and-restaurant-food/">telling food workers they don&#8217;t need face masks</A> does not exactly inspire confidence here.</p>
<p>I am really craving something other than the three or four things I can cook myself, and I have a lot of mutually-quarantined housemates to convince, so if any of you have any clearer estimate of the risk situation, please share.</p>
<p><b>Ventilator numbers</b></p>
<p>Britain <A HREF="https://www.reuters.com/article/us-health-coronavirus-britain-healthcare/who-gets-the-ventilator-british-doctors-contemplate-harrowing-coronavirus-care-choices-idUSKBN2172FC">has 5,000</A>, or one per 12,000 citizens. The US <A HREF="https://www.nbcnews.com/health/health-news/what-ventilator-critical-resource-currently-short-supply-n1168641">has 160,000</A>, or about 1 per 2,000 citizens (why are these numbers so different?). The head of a small ventilator company says they usually <A HREF="https://www.wired.com/story/ventilator-makers-race-to-prevent-a-possible-shortage/">&#8220;sell 50 in a good month&#8221;</A>.</p>
<p>Elon Musk recently <A HREF="https://www.massdevice.com/elon-musk-says-he-delivered-more-than-10000-ventilators-to-u-s/">delivered 1,255 ventilators</A> to California from some of Tesla&#8217;s Chinese contacts, and promised to make more. Dyson, the British vacuum manufacturer, says it will be able to <A HREF="https://www.ft.com/content/3a27f8f0-e0e7-4e5d-8760-b08669baee71">make 10,000 ventilators</A> in time to help with the crisis &#8211; remember, that&#8217;s twice what the whole UK has right now. The American Hospital Association <A HREF="https://www.medpagetoday.com/infectiousdisease/covid19/85462">says 960,000</A> Americans may require ventilators during the pandemic &#8211; hopefully not all at once. </p>
<p>Ventilators also require trained staff to operate. I never know how far to trust medical people when they say something requires training. You would think doing a lumbar puncture requires training, but the training I received for this in residency was watching one (1) guy do it one (1) time, and then them saying &#8220;Now you do it&#8221; &#8211; which by the way is exactly as scary as you would expect. This is an official thing in medical education, called <A HREF="https://www.ncbi.nlm.nih.gov/pmc/articles/PMC4785880/">see one, do one, teach one</A>. So when people say some medical task requires training, I don&#8217;t know if they mean &#8220;ten years&#8217; experience and a licensing exam&#8221;, &#8220;watch it once and then we throw you in the deep end&#8221; or &#8220;we&#8217;re going to make you go through the former, but the latter would have worked too&#8221;. Hopefully ventilators are more like the latter and someone can train new people <i>really</i> quickly.</p>
<p>If you&#8217;re confused about the difference between ventilators, oxygen concentrators, etc, or you have clever questions like &#8220;can we repurpose CPAP machines as ventilators?&#8221;, you might like Sarah Constantin&#8217;s <A HREF="https://srconstantin.github.io/2020/03/19/oxygen-supplementation-101.html">Oxygen Supplementation 101</A>.</p>
<p><b>The British reversal</b></p>
<p>A UK critical care doctor on Reddit wrote <A HREF="https://www.reddit.com/r/Coronavirus/comments/fnl0n6/im_a_critical_care_doctor_working_in_a_uk_high/#fla1iq6">a great explanation</A> of their recent about-face on coronavirus strategy.</p>
<p>They say that over the past few years, Britain developed a cutting-edge new strategy for dealing with pandemics by building herd immunity. It was actually really novel and exciting and they were anxious to try it out. When the coronavirus came along, the government plugged its spread rate, death rate, etc into the strategy and got the plan Johnson originally announced. This is why he kept talking about how evidence-based it was and how top scientists said this was the best way to do things.</p>
<p>But other pandemics don&#8217;t require ventilators nearly as often as coronavirus does. So the model, which was originally built around flu, didn&#8217;t include a term for ventilator shortages. Once someone added that in, the herd immunity strategy went from clever idea to total disaster, and the UK had to perform a disastrous about-face. Something something technocratic hubris vs. complexity of the real world.</p>
<p><b>Maybe we should have taken it easy with the huddled masses</b></p>
<p>China had Wuhan, Italy had Lombardy. Two weeks ago, everyone expected Seattle or the Bay would be the epicenter of the pandemic in the US. Well, right now both of those places combined have about 3,000 cases. New York City has 30,000. The New York/New Jersey area has about half the cases in the US, and is rising fast.</p>
<p>What changed? Partly the international epidemic shifted from Asia (which has immigrant communities and transportation links on the West Coast) to Italy and Europe (which have immigrant communities and transportation links on the East Coast). Partly the West Coast had some good policy whereas New York had terrible policy (while California was instituting shelter-in-place, Governor Cuomo was <A HREF="https://www.nydailynews.com/coronavirus/ny-coronavirus-20200317-ttcjmuqqvbdidn6yauz2huvpja-story.html">vetoing NYC&#8217;s shelter-in-place order</A> and later <A HREF="https://www.nationalreview.com/news/cuomo-slams-de-blasios-shelter-in-place-speculation-that-came-from-nuclear-war/">griping about the term shelter-in-place&#8217;s etymology</A>).</p>
<p>But the other major factor <A HREF="https://www.nytimes.com/2020/03/23/nyregion/coronavirus-nyc-crowds-density.html">seems to be density</A>. NYC is by far <A HREF="https://www.governing.com/gov-data/population-density-land-area-cities-map.html">the densest city in America</A>, almost twice as dense as second-placer San Francisco. Density forces people together and makes infections spread more easily.</p>
<p>At least that&#8217;s the story. So how come San Francisco &#8211; again, number two on the density list &#8211; has been almost completely spared? How come, despite its towering skyscrapers and close links to China, SF has only 178 diagnosed cases &#8211; fewer than such bustling metropolises as Indianapolis, Indiana, or Nashville, Tennessee? How come the virus is so well-behaved in very dense countries like Japan, and so deadly in relatively sparsely-populated places like Switzerland?</p>
<p>I&#8217;m not sure. Maybe density measures are really bad? Like if NYC annexed all of Long Island, it would drop to having one of the lowest densities in the nation on paper, but this purely political act wouldn&#8217;t affect its coronavirus susceptibility at all. Maybe there are enough problems like this that all existing density statistics average very dense areas with less dense areas and so don&#8217;t tell us what we want to know for disease spread.</p>
<p>Consider Spain. On paper, it&#8217;s one of the least densely-populated countries in Europe. In practice, it&#8217;s a lot of rolling countryside plus a few very dense cities &#8211; <A HREF="https://en.wikipedia.org/wiki/List_of_European_Union_cities_proper_by_population_density">four of the ten</A> densest cities in Europe are there. Maybe that&#8217;s why it&#8217;s got the fourth most cases in the world right now, behind only China, Italy, and the US?</p>
<p>The worst-affected US city <i>per capita</i> isn&#8217;t any of the ones I would have predicted &#8211; it&#8217;s New Orleans. Nathan Robinson lives there and <A HREF="https://www.currentaffairs.org/2020/03/the-virus-and-us">takes some guesses about why things there are so bad</A>. By the way, it&#8217;s going to reach 89 degrees in New Orleans tomorrow; keep that in mind whenever someone says the virus can&#8217;t spread in warm weather.</p>
<p><b>Irresponsible opinions on meds</b></p>
<p>Donald Trump <A HREF=https://twitter.com/realDonaldTrump/status/1241367239900778501">tweeted excitedly about</A> hydroxychloroquine/azithromycin, a drug combination which looked good in a single small preliminary trial against coronavirus but is otherwise unproven.</p>
<p>A few days later, an Arizona couple <A HREF="https://slate.com/news-and-politics/2020/03/arizona-man-dies-chloroquine-trump-coronavirus-advice.html">took a fish-tank cleaner including the closely-related drug chloroquine</A> to try to protect themselves from the disease. The man died and the woman is in the ICU.</p>
<p>First things first &#8211; from a medical perspective, what went wrong here? Fish tank chloroquine is chloroquine phosphate, which is a perfectly acceptable form of chloroquine approved for human consumption as the antimalarial drug Aralen. Chloroquine has lots of nasty side effects, but none of them are bad enough to kill you instantly. My guess is that the guy either took orders of magnitude too high a dose &#8211; the news articles just say &#8220;a spoonful&#8221; &#8211; or that there were other things in the fish tank cleaner. Interested to hear from doctors who know more about chloroquine on this.</p>
<p>Okay, now let&#8217;s get to the controversial part: is Trump responsible? He seems causally responsible, in the sense that his endorsement led to the overdose. But is he <i>morally</i> responsible? I just got done telling all of you that stop-smoking-aid bupropion is an amazing drug that can save your life. If one if you is an idiot and responds by taking 100 times the safe dose of some industrial chemical with bupropion in it, does that make me responsible for your death? Is the difference that bupropion is known to work, but chloroquine is only speculative? Why should this change how we distribute responsibility?</p>
<p>Maybe responsibility is the wrong lens here? Maybe Presidents should be aware that they have such an immense platform that all of their statements can be interpreted in absurd ways, and perform a cost-benefit analysis before saying anything at all? Maybe (to go back to my example), the cost benefit analysis passes muster for bupropion, because the chance that one of you does something idiotic and kills yourself is counterbalanced by the chance that many of you use it correctly and stop smoking? But responsible scientists were going to investigate hydroxychloroquine responsibly before Trump said anything, so his statement had no benefit and he should have thought more about the costs.</p>
<p>I appreciate this line of reasoning, but I hate it. It means you stop being able to communicate your real thoughts in favor of communicating whatever information a utility calculation says it&#8217;s most beneficial to communicate &#8211; which is fine until people very reasonably choose to stop interpreting your mouth movements as words. On the other hand, the President of the US is not really supposed to be a clearinghouse for medical information, and is definitely somebody whose words have direct effects on the world, so maybe we should make an exception for him.</p>
<p>For a fun example of how complicated this way of thinking becomes, @WebDevMason <A HREF="https://twitter.com/webdevMason/status/1242424349849579520">condemns the media for over-reporting on fish-tank-man&#8217;s death</A>. She points out that that hydroxychloroquine may yet prove effective and become an important part of our arsenal against coronavirus. And when doctors start trying to prescribe it, a big chunk of the US population is going to know it only as &#8220;that thing Trump irresponsibly recommended even though it&#8217;s an ingredient in fish tank cleaners that kills you if you take it&#8221;. And they&#8217;re going to freak out and refuse. Might this also cause deaths? Who knows! </p>
<p>So who deserves blame here? Trump, for irresponsibly praising the drug? The media, for irresponsibly condemning Trump for praising the drug? Mason, for irresponsibly condemning the media for condemning Trump for praising the drug? Me, for irresponsibly praising Mason for condemning the media for condemning Trump for praising the drug? <s>Had gadya, had gadya</s>.</p>
<p><b>The third world</b></p>
<p>&#8230;is in really deep trouble, isn&#8217;t it?</p>
<p>The numbers say it isn&#8217;t. Less developed countries are doing fine. Nigeria only has 65 cases. Ethiopia, 12 cases. Sudan only has three! </p>
<p>But they probably just aren&#8217;t testing enough. San Diego has 337 diagnosed cases right now. The equally-sized Mexican city of Tijuana, so close by that San Diegans and Tijuanans <A HREF="https://en.wikipedia.org/wiki/San_Diego%E2%80%93Tijuana#Sports">play volleyball over the border fence</A>, has 10. If we assume that the real numbers are more similar (can we assume this?), then Mexico is undercounting by a factor of 30 relative to the US, which is itself undercounting by a factor of 10 or so. This would suggest Mexico has the same number of cases as eg Britain, which doesn&#8217;t seem so far off to me (Mexico has twice as many people).</p>
<p>The developing world doesn&#8217;t have many ventilators and doesn&#8217;t have enough state capacity to enforce self-isolation very effectively. It&#8217;s full of very densely packed slums. It has a lot of dictators who like to deny the existence of problems and shoot anyone who keeps insisting a problem exists. It could get really bad.</p>
<p>I worry that nobody has the spare energy to do anything about this. The First World is busy saving itself. Rich countries will probably corner the facemask and ventilator supply. The kind of doctors who go to Doctors Without Borders are probably at home busy saving their countrymen. Everyone else is going to have <i>such</i> a bad time, with few reasons for optimism.</p>
<p>I&#8217;m not even sure what concerned people can do. Charities&#8217; usual MO is to divert resources from First World countries to Third World ones, but First World countries are using all their relevant resources and won&#8217;t sell for any price &#8211; can you imagine trying to export ventilators from the US right now? You&#8217;d probably get arrested. Maybe the highest-leverage interventions are figuring out how to repurpose cheap pre-existing material for medical care &#8211; face masks made out of paper/cloth/whatever, ventilators out of ???.</p>
<p>Nigeria and Mexico and so on make me confused in the same way as Japan &#8211; why aren&#8217;t they already so bad that they can&#8217;t hide it? If the very poorest countries in sub-Saharan Africa were suffering a full-scale coronavirus epidemic, would we definitely know? In Liberia, only 3% of people are aged above 65 (in the US, it&#8217;s 16%). It only has one doctor per 100,000 people (in the US, it&#8217;s one per 400) &#8211; what does &#8220;hospital overcrowding&#8221; even mean in a situation like that? I don&#8217;t think a full-scale epidemic could stay completely hidden forever, but maybe it could be harder to notice we would naively expect.</p>
<p><b>How can you help?</b></p>
<p>Sanjay on the Effective Altruism forum has a post about <A HREF="https://forum.effectivealtruism.org/posts/wpaZRoLFJy8DynwQN/the-best-places-to-donate-for-covid-19">the best places to donate [to fight] COVID-19</A>. Some of these are long-term work of questionable immediate relevance &#8211; the Johns Hopkins Center on Health Security does great work, but I wonder if a donation now just means that they hire some better researchers in six months and produce better policy recommendations next year. Also, I predict biosecurity think tanks won&#8217;t be funding-constrained for the immediate future.</p>
<p><A HREF="https://www.developmentmedia.net/">Development Media International</A> and <A HREF="https://www.univursa.com/">Univursa Health</A> are their recommendations for where to donate to help fight coronavirus in the Third World, but as far as I can tell neither organization is publicly doing that yet &#8211; they just seem like the kind of organizations that could and will eventually have to.</p>
<p>The writer is not entirely sure you should donate to coronavirus control at all &#8211; everyone&#8217;s doing it and the field probably has enough funding to pick most low-hanging fruits. Remember that malaria still kills 400,000 people per year (about 20% of the <A HREF="https://www.metaculus.com/questions/3530/how-many-people-will-die-as-a-result-of-the-2019-novel-coronavirus-covid-19-before-2021/">expected coronavirus death toll</A>) but is probably getting a tiny fraction of the funding and attention right now.</p>
<p>Give Directly, previously known for giving cash directly to poor Africans, is now also working on giving cash directly to Americans who are affected by coronavirus. You can read about their program <A HREF="https://www.vox.com/future-perfect/2020/3/20/21186007/coronavirus-pandemic-donate-help-cash-benefits">here</A>, and donate <A HREF="https://www.givedirectly.org/covid-19/">here</A>.</p>
<p>The Frontline Responders Fund is working with Silicon Valley logistics company Flexport to try to transport masks and other medical supplies from producers to people who need them. You can read about them <A HREF="https://techcrunch.com/2020/03/24/flexport-arnold-schwarzenegger-and-others-launch-a-fund-to-get-supplies-to-frontline-responders/">here</A> and donate <A HREF="https://www.gofundme.com/f/frontlinerespondersfund">here</A>. </p>
<p><A HREF="https://www.reddit.com/r/CoronavirusArmy/">r/CoronavirusArmy</A> is the subreddit for people trying to coordinate various useful virus response projects. There&#8217;s the expected massive variation in quality, but some of them could be really helpful.</p>
<p><b>Worth it</b></p>
<p>A lot of people are secretly wondering whether preventing the potential damage from coronavirus is really worth shutting down the entire economy for months. You shouldn&#8217;t feel ashamed for wondering that. Everyone, <A HREF="https://en.wikipedia.org/wiki/Value_of_life">including the US government</A>, agrees that it is sometimes worth putting a dollar cost on human life, and there are all sorts of paradoxes and ridiculous behaviors you get trapped in if you refuse to do so.</p>
<p>Some people on <A HREF="https://www.reddit.com/r/slatestarcodex/comments/fmhk43/cost_of_saving_life_years_that_would_be_lost_due/fl4xhg8/">this thread on the subreddit</A> have tried to calculate it out, using the government&#8217;s value-of-one-life-year figures. There are a lot of variables involved that we can only guess at, but given some reasonable predictions, even at a low value of $30,000 per life-year it&#8217;s worth spending trillions of dollars to slow down the epidemic.</p>
<p>I don&#8217;t want anyone to feel uncomfortable discussing this, so if you disagree or have different calculations please feel like the comments here are a safe place to talk about it.</p>
<p><b>But no, You sent us Congress</b></p>
<p>The Senate mercifully approved a stimulus bill earlier this week. I say &#8220;mercifully&#8221; because watching the negotiations was painful. I still have no idea which party was Boldly Trying To Provide The American People With Necessary Relief and which one was trying to hold the bill hostage in order to add a wish list of stupid partisan demands.</p>
<p>The narrative I&#8217;ve been hearing from Democrats was that they were Boldly Trying To Provide The American People With Necessary Relief by giving loans to nonprofits, and the Republicans held it hostage by hamhandedly adding rules intended to guarantee that <A HREF="https://www.washingtonpost.com/opinions/2020/03/23/gop-just-smuggled-another-awful-provision-into-big-stimulus-bill/">none of the loans could go to Planned Parenthood in particular</A> &#8211; hamhandedly because the particular fig leaf they used &#8211; &#8220;no loans to nonprofits receiving Medicaid funding&#8221; &#8211; also disqualifies anyone else who helps poor people get healthcare.</p>
<p>The narrative I&#8217;ve been hearing from Republicans was that they were Boldly Trying To Provide The American People With Necessary Relief by giving loans and money to a broad selection of the American people, and the Democrats held it hostage by trying to make it about all of their pet issues instead. So <A HREF="https://www.nationalreview.com/corner/congressional-democrats-add-last-minute-ideological-demands-to-coronavirus-relief-package/">National Review</A> makes fun of Democratic demands that the package include rules restricting carbon emissions and expanding the bargaining power of unions. (see conservative satire site <A HREF="https://babylonbee.com/news/dems-demand-stimulus-bill-include-reparations-for-transgender-native-americans-affected-by-climate-change">Babylon Bee</A> for the complete list, YES I KNOW THIS IS FAKE). </p>
<p>But apparently all that got cleared up, and now the bill is under threat from &#8211; libertarians! According to <A HREF="https://www.washingtonpost.com/politics/house-leaders-seek-to-expedite-emergency-aid-package-amid-uncertainty-about-gop-lawmaker-delaying-measure/2020/03/26/392c9dba-6f7d-11ea-a3ec-70d7479d83f0_story.html">the Washington Post</A>, the main holdout in the House of Representatives is a &#8220;constitutional libertarian&#8221; who&#8217;s  trying to prevent the House from voting remotely because the Constitution says it shouldn&#8217;t.</p>
<p>I have a lot of respect for principled constitutionalists who believe that the nation&#8217;s government should occasionally follow the document that they take a solemn oath to protect. But insisting on that <i>now</i>, of all times, seems kind of like closing the barn door after the horse has left, caught a plane to Cape Canaveral, boarded an experimental rocketship, gotten halfway to the Oort Cloud, and also some kind of weird terrorist group is threatening to start a nuclear apocalypse if anyone closes any barn doors. Just let this one go and get back to your noble-yet-quixotic crusade sometime when we&#8217;re not all going to die.</p>
<p><b>Getting it Right</b></p>
<p>It took the mainstream media a while to realize the seriousness of the coronavirus. The right wing has its own parallel media system, and I&#8217;ve heard accusations that it failed even worse, and for longer. I can&#8217;t comment on whether this was true at the time, but it seems to have improved; as of me writing this, fox.com, breitbart.com, nationalreview.com, and townhall.com all have front pages full of the same kind of frantic coronavirus news I would expect to see anywhere else. Reddit&#8217;s big pro-Trump subreddit <A HREF="https://www.reddit.com/r/The_Donald/">r/The_Donald</A> has a sticky thread of &#8220;President Trump&#8217;s Coronavirus Guidelines For America &#8211; 15 Days To Stop The Spread &#8211; READ AND FOLLOW&#8221;. Even the front page of Infowars urges readers <A HREF="https://www.infowars.com/meet-the-covidiots-new-word-invented-for-people-who-ignore-coronavirus-danger/">Don&#8217;t Be A Covidiot</A> &#8211; their term for someone who ignores the danger of coronavirus and doesn&#8217;t practice good social distancing.</p>
<p>Still, that was the result of a long battle. Just like on the left, a few prescient right-wingers had to battle to make their friends and colleagues realize the danger. I&#8217;ve heard <A HREF="https://www.latimes.com/entertainment-arts/business/story/2020-03-21/fox-news-host-tucker-carlson-on-why-he-sounded-the-alarm-on-the-coronavirus">Tucker Carlson deserves special honor</A> for fighting the good fight when the rest of FOX was trying to downplay everything. <A HREF="https://www.bbc.com/news/world-us-canada-52009108">Steve Bannon and Lindsey Graham</A> also took a hard line and helped their colleagues see reason.</p>
<p>I&#8217;m not sure what the role of liberals (here used as a general term encompassing everyone except the hard right) should be in this process. I can only beg us not to mess it up. Calling right-wingers dumb for not getting the point fast enough risks messing it up; it could just make them more stubborn and angry. Also, Trump is the acknowledged world expert at reaching Trump supporters. If he thinks that calling it &#8220;the Chinese virus&#8221; will convince his xenophobic fans to take it seriously, consider not messing with that.</p>
<p><b>Short Links</b></p>
<p>Iceland has finally done what everyone&#8217;s been waiting for and <A HREF="https://english.alarabiya.net/en/features/2020/03/25/Coronavirus-Iceland-s-mass-testing-finds-half-of-carriers-show-no-symptoms">tested lots of people to see how many are asymptomatic</A>. They conclude that about half of carriers don&#8217;t know they have the disease. If there had been very many more asymptomatic carriers than symptomatic patients, it would have been good news &#8211; most cases never show up in the statistics, and all of our estimates of hospitalization rate and mortality rate would be much too high. Although it&#8217;s nice to be able to divide all of those by two, a lot of people were hoping we could divide them by ten or a hundred and stop worrying completely. This study suggests we can&#8217;t. [EDIT: <A HREF="https://slatestarcodex.com/2020/03/27/coronalinks-3-27-20/#comment-871099">jgr79 points out</A> a more optimistic interpretation: the testing happened around March 20, when Iceland had 300 reported cases, but detected that 1% of Icelanders were positive, ie 3,000 reported cases. This matches <A HREF="https://slatestarcodex.com/2020/03/19/coronalinks-3-19-20/">all the other evidence</A> that real cases outnumber diagnosed cases by a factor of 10 or so, and probably does mean we can divide observed mortality rates by that amount. Is everyone already doing this in their models?]</p>
<p>In 1918, people got so tired of containment procedures for the Spanish Flu that concerned citizens started an <A HREF="https://twitter.com/Scholars_Stage/status/1242084301719683072">Anti-Mask League Of San Francisco</A>.</p>
<p>Future of Humanity Institute has put up a <A HREF="http://epidemicforecasting.org/">dashboard making advanced pandemic modelling software available to the public</A>. They are also also offering pro bono forecasting services to under-resourced groups like hospitals and governments in developing nations). They&#8217;ve asked me to help spread the word on this, and I will, but I&#8217;d be more comfortable if someone who knows their stuff can confirm it&#8217;s net helpful, so please contact me if you consider yourself informed enough to have an opinion on this.</p>
<p>Robin Hanson <A HREF="http://www.overcomingbias.com/2020/03/know-when-to-fold-em.html">makes the case for variolation</A> &#8211; deliberately exposing people to virus particles at low doses through routes that make the infection less dangerous. This operates as kind of a poor man&#8217;s vaccine, giving a very mild case that prevents the person from getting sick in the future. Has worked with many past epidemics (like smallpox), still unknown how to predict how it would work for this one.</p>
<p>Hall of shame: Bangladesh (where <A HREF="https://www.thestar.com.my/news/world/2020/03/19/25000-people-gather-for-covid-19-prayer-session-in-bangladesh-sparking-outcry">25,000 people have gathered</A> for a mass prayer rally against the coronavirus &#8211; if only the New Atheists were still around to offer opinions on this kind of thing). Mississippi, as usual (see <A HREF="https://www.reddit.com/r/Coronavirus/comments/fo43p5/governor_rejects_state_lockdown_for_covid19/fld6q2t/">this comment</A> by an MS Redditor). Russia, <A HREF="https://www.codastory.com/waronscience/russia-coronavirus-mistrust/">as usual</A>. Donald Trump is a <A HREF="https://twitter.com/realDonaldTrump/status/1242905328209080331">permanent lifetime member</A> at this point. The FDA is also probably a <A HREF="https://www.wsj.com/articles/trump-sought-to-expand-virus-drug-tests-over-fda-objections-11584545251">permanent lifetime member</A>. </p>
<p>Last links post I included tech company Triplebyte in the shame list for refusing to let employees switch to work-from-home, then firing them. A representative of Triplebyte contacted me and asked me to explain their perspective, which is that they took the pandemic seriously and went all-remote around the same time as everyone else. The reluctance to let employees switch to work-from-home applied only to a few employees in early March, before the scale of the crisis was widely appreciated, and they say that they would have tried to make accommodations if they had understood the seriousness of the requests. They had been planning the downsizing for a while, it was really unlucky that it ended up in the middle of a pandemic, and they tried to make it as painless as possible by offering good severance pay, etc. I&#8217;m relaying their statement because I&#8217;m realizing it was probably unfair of me to single them out in particular &#8211; my hearing a lot about this was downstream of my having a lot of friends who work(ed) for Triplebyte, and my having a lot of friends who work(ed) for Triplebyte was downstream of them being a great company doing important work which all my friends wanted to work for. I continue to generally respect them and their vision (see <A HREF="https://slatestarcodex.com/2017/07/24/targeting-meritocracy/">here</A> for more), and you don&#8217;t need to give them any more grief over it than they&#8217;re already getting.</p>
<p>Hall of fame: Service Employees International Union (&#8220;found&#8221; <A HREF="https://www.nbcbayarea.com/news/coronavirus/seiu-locates-39-million-n95-masks-for-healthcare-workers-local-governments/2262072/">40 million face masks</A> and is donating them to local hospitals; what does it even mean to &#8220;find&#8221; this many masks in this context?), and Amazon (now giving workers <A HREF="https://www.theverge.com/2020/3/21/21189333/amazon-pay-warehouse-workers-double-time-coronavirus">double pay for overtime</A>). And Brazilian gangs, in the face of government inaction, <A HREF="https://twitter.com/AndrewCesare/status/1242174265547468803">declared a unilateral quarantine order in Rio de Janeiro</A>, saying &#8220;If the government won&#8217;t do the right thing, organized crime will&#8221;. I deeply appreciate the commentator who <A HREF="https://twitter.com/JeffJMason/status/1242284934120734721">described</A> this as &#8220;state capacity anarcho-capitalism&#8221;.</p>}
\end{xmlentriescontent}
\xmlentrieswfwcommentrss{https://slatestarcodex.com/2020/03/27/coronalinks-3-27-20/feed/}\xmlentriesslashcomments{954}
\end{xmlentries}
\xmletag{W/"dcd5f4f0bec25fbeefc590bab57583f7-gzip"}\xmlupdated{Fri, 10 Apr 2020 15:46:25 GMT}\xmlhref{https://slatestarcodex.com/feed/}\xmlencoding{UTF-8}\xmlversion{rss20}